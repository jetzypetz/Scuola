\documentclass{article}

\usepackage{tikz}
\usepackage{parskip}
\usepackage{amssymb}
\usepackage[utf8]{inputenc}
\usepackage{amsmath, empheq}
\usepackage[margin=3.75cm]{geometry}
\definecolor{mycolour}{RGB}{46,52,64}
\pagecolor{mycolour}
\color{white}

\title{\jobname}
\author{Eugenio Animali}

\begin{document}
\maketitle

\section{Progressioni}
\subsection{Successioni}
\begin{gather*}    
    f:\mathbb{N} \to \mathbb{R}
\end{gather*}

Si esprimono in tre modi:
\begin{gather}
    a_n=3n+2\\
    2,5,8,11,\dots\\
    \begin{cases}
        a_0=2\\
        a_n=a_{n-1}+3
    \end{cases}
\end{gather}

Somma di tutti i valori fino ad un limite:
\[S_n=(a_1+a_n)\frac{n}{2}\]
\begin{enumerate}
    \item Aritmetiche
    
    la ragione é un valore che viene aggiunto ogni volta:
    \begin{gather*}
        2,4,6,8,10,\dots\\
        \begin{cases}
            a_0=0\\
            r=2\\
            a_n=a_{n-1}+r
        \end{cases}
    \end{gather*}
    \item Geometriche
    \begin{gather*}
        10,20,40,80,\dots\\
        \begin{cases}
            q=2\\
            k=10\\
            a_n=k\times q^{n-1}
        \end{cases}
    \end{gather*}
\end{enumerate}
\subsection{Principio di Induzione}
Mi viene data una affermazione:
\[
    3+6+9+\dots +3n=\frac{3}{2}n(n+1)
    \]
    Devo dimostrare che (1) é vero per $n=1$, e che (2) se é vero per $n$, é anche vero per $n+1$ e cosí dimostro che é valido per tutti i valori naturali di $n$.
    \begin{gather*}
    3=\frac{3}{2}\cdot 1(1+1)\\
    3=\frac{3\cdot 2}{2}\text{ ACC}\\
\end{gather*}
Ora considero vero che $3+6+9+\dots +3n=\frac{3}{2}n(n+1)$ e studio per $n+1$
\begin{gather*}
    \frac{3}{2}n(n+1)+3(n+1)=\frac{3}{2}(n+1)(n+2)\\
    \frac{3}{2}n+3=\frac{3}{2}(n+2)\\
    \frac{3}{2}n+\frac{3}{2}\cdot\frac{2}{3}3=\frac{3}{2}(n+2)\\
    \frac{3}{2}n+\frac{3}{2}2=\frac{3}{2}(n+2)\\
    n+2=n+2\\
    n=n\text{ ACC}
\end{gather*}
\end{document}