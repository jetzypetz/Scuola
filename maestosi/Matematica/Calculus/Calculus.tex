\documentclass{article}
\usepackage[utf8]{inputenc}
\usepackage{amsmath, empheq}
\usepackage[dvipsnames]{xcolor}
\usepackage[margin=3.75cm]{geometry}
\definecolor{mycolour}{RGB}{46,52,64}
\pagecolor{mycolour}
\color{white}
\title{Calculus}
\author{Eugenio Animali}
\date{4 sept 2022}
\begin{document}
\maketitle
\section*{That famous Oxymoron}
'Instantaneous rate of change'\\
In calculus, we do not measure an instantaneous rate of change; we measure what a rate approaches, with dx approaching 0. Seeing what this rate approaches shows what rate is tangent to the curve.
\section{Derivative}
\begin{enumerate}
    \item
Here is the general equation for the derivative of a point on a function. It shows the rise over run using a value of dx which approaches 0
\[
\frac{df(x)}{dx} = \frac{f(x +dt) - f(x)}{dx}
\]
\begin{gather*}
\label{example 1} \text{To work through an example, let's take the derivative of } y=x^{3}\text{ while }x = 4:\\
    \lim_{dx\to 0} \frac{df(x)}{dx}=\lim_{dx\to 0} \frac{(x+dx)^2-x^2}{dx}\\
    =\lim_{dx\to 0} \frac{x^2+dx^{3}-x^2}{dx}\\
    =\lim_{dx\to 0} 3x^{2}+3xdx+dx^{2}\\
    \text{as dx approaches 0, all parts with dx as a coefficient will approach 0:}\\
    =3x^2
\end{gather*}
    \item Let's find a general formula for derivatives of polynomials:\\
    $$f(x) = x^{n}$$
\begin{gather*}
    (x+dx)^n =\overbrace{(x+dx)(x+dx)(x+dx)\dots (x+dx)}^{n\text{ times}}\\
    =x^n+n\color{green}x^{n-1}\color{red}dx\color{white}\dots+\text{Multiples of }dx^2
\end{gather*}
\[\frac{dy}{dx} = \frac{(x+dx)^{n}-x^n}{dx}\]
\end{enumerate}
\end{document}