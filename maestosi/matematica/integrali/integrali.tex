\documentclass{article}

\usepackage{tikz}
\usepackage{parskip}
\usepackage{amssymb}
\usepackage[utf8]{inputenc}
\usepackage{amsmath, empheq}
\usepackage[margin=3.75cm]{geometry}
\definecolor{mycolour}{RGB}{46,52,64}
\pagecolor{mycolour}
\color{white}

\title{\jobname}
\author{Eugenio Animali}

\begin{document}
\maketitle

\section{Integrali indefiniti}

\begin{gather*}
    \int f(x)dx = F(x) + c
\end{gather*}
\section{Integrali elementari}
\begin{gather*}
    \int dx = x + c\\
    \int x^\alpha dx = \frac{x^{\alpha + 1}}{\alpha + 1} + c \text{ eccetto } \alpha = - 1\\
    \int \frac{1}{x} dx = \ln |x| + c\\
    \int \sin x dx = - \cos x + c\\
    \int \frac{1}{\cos^2 x}dx = \tan x + x\\
    \int \frac{1}{\sqrt{1 - x^2}} dx = \arcsin x + c
\end{gather*}

\section{Integrale della Funzione Composta}
Se nella funzione integranda trovo una funzione e la sua derivata, posso usare la regola della funzione composta.
Sapendo che il differenziale é
\begin{gather*}
    df(x) = f'(x) dx
\end{gather*}
quindi:
\begin{gather*}
    \int f(x) f'(x) dx = \int f(x) d(f(x)) = \frac{f^2(x)}{2} + c
\end{gather*}
Ció vale anche se $f(x)$ é interna ad una funzione $g(f(x))$ piú complessa. Basta manipolare la funzione integranda finché presenta una $f(x)$ che corrisponda ad una $f'(x)$.
\section{Integrazione per Sostituzione}
Particolarmente utile nei seguenti casi:
\begin{enumerate}
    \item Radici
    \item Esponenziali
    \item Goniometria
\end{enumerate}
Il metodo consiste nel sostituire x per un'altra funzione $t$ scelta da me, per togliere parti complicate della funzione. per passare da $x$ a $t$, devo considerare l'effetto ché avrá su dx:
\begin{gather*}
    x = g(t)\\
    dx = g'(t)dt
\end{gather*}

Esempio:
\begin{gather*}
		\int \frac{x}{\sqrt{x^2 - 9}}dx
\end{gather*}
Pongo:
\begin{gather*}
		t=\sqrt{x^2 - 9}\\
		x = \sqrt{t^2 + 9}\\
		g'(t) = \frac{1}{2\sqrt{t^2 + 9}}\cdot 2t =\frac{t}{\sqrt{t^2+9}}\\
		dx=\frac{t}{\sqrt{t^2+9}}dt
\end{gather*}
Quindi:
\begin{gather*}
		\int \frac{x}{\sqrt{x^2-9}}dt=\int \frac{\sqrt{t^2+9}}{t}\frac{t}{\sqrt{t^2+9}}dt\\
		= t +c =\sqrt{x^2-9}+c
\end{gather*}
\section{Integrazione per Parti}
Quando ho una funzione facilmente derivabile e una facilmente integrabile.
\begin{gather*}
		\int f(x)\cdot g'(x)dx = f(x)\cdot g(x) - \int f'(x)\cdot g(x)dx
\end{gather*}
Esempio:
\begin{gather*}
		\int xe^x dx		
\end{gather*}
Pongo:
\begin{gather*}
		f(x) = x, f'(x) = 1\\
		g(x) = e^x, g'(x) = e^x\\
\end{gather*}
Quindi:
\begin{gather*}
		xe^x - \int e^x dx = xe^x - e^x + c
\end{gather*}
\end{document}
