\documentclass{article}

\usepackage{tikz}
\usepackage{parskip}
\usepackage{amssymb}
\usepackage[utf8]{inputenc}
\usepackage{amsmath, empheq}
\usepackage[margin=3.75cm]{geometry}
\definecolor{mycolour}{RGB}{46,52,64}
\pagecolor{mycolour}
\color{white}

\title{\jobname}
\author{Eugenio Animali}

\begin{document}
\maketitle

\section{Integrali indefiniti}

\begin{gather*}
    \int f(x)dx = F(x) + c
\end{gather*}
\section{Integrali elementari}
\begin{gather*}
    \int dx = x + c\\
    \int x^\alpha dx = \frac{x^{\alpha + 1}}{\alpha + 1} + c \text{ eccetto } \alpha = - 1\\
    \int \frac{1}{x} dx = \ln |x| + c\\
    \int \sin x dx = - \cos x + c\\
    \int \frac{1}{\cos^2 x}dx = \tan x + x\\
    \int \frac{1}{\sqrt{1 - x^2}} dx = \arcsin x + c\\
	\int a^x dx = \frac{1}{\ln a}a^x + c
\end{gather*}

\section{Integrale della Funzione Composta}
Se nella funzione integranda trovo una funzione e la sua derivata, posso usare la regola della funzione composta.
Sapendo che il differenziale é
\begin{gather*}
    df(x) = f'(x) dx
\end{gather*}
quindi:
\begin{gather*}
    \int f(x) f'(x) dx = \int f(x) d(f(x)) = \frac{f^2(x)}{2} + c
\end{gather*}
Ció vale anche se $f(x)$ é interna ad una funzione $g(f(x))$ piú complessa. Basta manipolare la funzione integranda finché presenta una $f(x)$ che corrisponda ad una $f'(x)$.
\section{Integrazione per Sostituzione}
Particolarmente utile nei seguenti casi:
\begin{enumerate}
    \item Radici
    \item Esponenziali
    \item Goniometria
\end{enumerate}
Il metodo consiste nel sostituire x per un'altra funzione $t$ scelta da me, per togliere parti complicate della funzione. per passare da $x$ a $t$, devo considerare l'effetto ché avrá su dx:
\begin{gather*}
    x = g(t)\\
    dx = g'(t)dt
\end{gather*}

Esempio:
\begin{gather*}
		\int \frac{x}{\sqrt{x^2 - 9}}dx
\end{gather*}
Pongo:
\begin{gather*}
		t=\sqrt{x^2 - 9}\\
		x = \sqrt{t^2 + 9}\\
		g'(t) = \frac{1}{2\sqrt{t^2 + 9}}\cdot 2t =\frac{t}{\sqrt{t^2+9}}\\
		dx=\frac{t}{\sqrt{t^2+9}}dt
\end{gather*}
Quindi:
\begin{gather*}
		\int \frac{x}{\sqrt{x^2-9}}dt=\int \frac{\sqrt{t^2+9}}{t}\frac{t}{\sqrt{t^2+9}}dt\\
		= t +c =\sqrt{x^2-9}+c
\end{gather*}
\section{Integrazione per Parti}
Quando ho una funzione facilmente derivabile e una facilmente integrabile.
\begin{gather*}
		\int f(x)\cdot g'(x)dx = f(x)\cdot g(x) - \int f'(x)\cdot g(x)dx
\end{gather*}
Esempio:
\begin{gather*}
		\int xe^x dx		
\end{gather*}
Pongo:
\begin{gather*}
		f(x) = x, f'(x) = 1\\
		g(x) = e^x, g'(x) = e^x\\
\end{gather*}
Quindi:
\begin{gather*}
		xe^x - \int e^x dx = xe^x - e^x + c
\end{gather*}
\section{Integrale con Frazione di Polinomi}
Posso separare la mia frazione in parti per renderla piu facile da gestire
\begin{gather*}
		\frac{x^2+3x+2}{(x+2)(x-3)}=\frac{A}{x+2}+\frac{B}{x-3}
\end{gather*}

\subsection{Integrale con Denominatore con $\Delta$ = 0}
Se il denominatore e un quadrato, si separa in parti nel seguente modo:
\begin{gather*}
		\int \frac{x + 5}{x^2-6x+9}dx\\
		\frac{x+5}{(x-3)^2} = \frac{A}{x-3}+\frac{B}{(x-3)^2}\\
	=\frac{A(x-3)+B}{(x-3)^2}\\
	\frac{x+5}{(x-3)^2}=\frac{Ax-3A+B}{(x-3)^2}
\end{gather*}
Visto che sto cercando A e B non e un problema dividere per $(x-3)^2$ e risolvere. Cosi separo il problema in parti piu semplici.
\subsection{$\Delta$ negativo}
Il primo obbiettivo e di avere al denominatore solo un numero.
\begin{gather*}
		\int \frac{3x+1}{x^2-4x+9}dx
\end{gather*}
prima provo a trasformare il numeratore per diventare la derivata del denominatore:
\begin{gather*}
		\frac{d(x^2-4x+9)}{dx}=2x-4\\
\end{gather*}
Quindi posso usare la prima regola di integrazione (funzione composta):
\begin{gather*}
		\int \frac{3(x+\frac{1}{3})}{x^2-4x+9}dx=\frac{3}{2}\int \frac{2(x+\frac{1}{3})}{x^2-4x+9}dx\\
		=\frac{3}{2} \int \frac{2x+\frac{2}{3}-4+4}{x^2-4x+9}dx=\frac{3}{2}\left[ \int \frac{2x-4}{x^2-4x+9}dx+ \int \frac{\frac{14}{3}}{x^2-4x+9}dx\right]\\
		=\frac{3}{2}\left[ \ln (x^2-4+9)+\frac{14}{3}\int \frac{1}{x^2-4x+9}dx\right]
\end{gather*}
Ora per risolvere la seconda parte, posso evidenziare il quadrato del binomio:
\begin{gather*}
	\int \frac{1}{x^2-4x+9}dx = \int \frac{1}{(x^2-4x+4)+5}dx\\
	=\int \frac{1}{(x-2)^2+5}=\frac{1}{5}\int \frac{1}{\frac{(x-2)^2}{5}+1}dx\\
=\frac{1}{5} \int \frac{1}{\left( \frac{x-2}{\sqrt{5}}\right) + 1}
\end{gather*}
con:
\begin{gather*}
		t=\frac{x-2}{\sqrt{5}}\\
		\int \frac{1}{1+t^2}dt = \arctan t
\end{gather*}
studiamo:
\begin{gather*}
		dx=\sqrt{5}dt
\end{gather*}
Quindi:
\begin{gather*}
		\frac{1}{5}\int \frac{1}{t^2+1}\sqrt{5}dt
\end{gather*}
\subsection{polinomi alla 3za potenza}
Utilizza ruffini per trovare fattori del denominatore e separa cosi il problema:
\begin{gather*}
		\frac{A}{fattore_{1}}+\frac{Bx +C}{fattore_{2}}
\end{gather*}
dove fattore 2 sta alla 2da potenza
\section{Integrali Definiti}
il problema delle aree e quello di determinare l'area sotto una curva delimitata. per risolverlo posso dividere l'intervallo $x$ in tanti $dx$ e fare una somma infinita:
\begin{gather*}
		\int_a^b f(x)dx
\end{gather*}
\subsection{teorema di torricelli-barrow}
data una integrale definita in questo modo:
\begin{gather*}
		F(x)=\int_a^x f(t)dt
\end{gather*}
allora la sua derivata e:
\begin{gather*}
		F'(x)=f(x)
\end{gather*}
vediamo:
\begin{gather*}
		F(a)=0+c\\
		F(b)=\int_a^b f(x)dx +c\\
		\int_a^b f(x)dx=F(b)-F(a)
\end{gather*}
Esempio:
\begin{gather*}
		\int_4^9 (3\sqrt{x}+2x)dx=\left[2x^{\frac{3}{2}}\right]_4^9\\
		F(b)-F(a)
\end{gather*}

\section{valore medio del grafico}

l'area sotto al grafico tra $a$ e $b$, anche per una curva, sara uguale all'area di un rettangolo con base $ab$, e altezza che arriva in un qualche punto della curva tra $a$ e $b$

\section{area tra due curve}

\begin{gather*}
A=\int_a^b f(x)dx - \int_a^b g(x)dx = \int_a^b \left[ f(x) - g(x)\right]dx
\end{gather*}

per trovare l'area tra varie curve, si segue la linea di contorno della forma, aggiungendo o sottraendo le integrali definite in base a se la linea segue verso destra o verso sinistra

\section{solidi di rotazione}

ruotando una curva attorno all'asse x, si crea una forma tridimenzionale, la cui area si calcola considerando non piu piccoli rettangoli, ma circonferenze con raggio $f(x)$, e quindi area  $\pi f(x)^2 dx$. l'integrale  $A = \pi \int [f(x)]^2 dx$ trova l'area completa del solido.

l'area della sfera si trova con il solido di rotazione di  $y=\sqrt{r^2-x^2}$ 

attorno alla y: vai a rivedere. $A =2 \pi \int [x f(x)] dx$

\end{document}
