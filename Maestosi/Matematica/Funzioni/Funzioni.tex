\documentclass{article}

\usepackage{tikz}
\usepackage{parskip}
\usepackage{amssymb}
\usepackage[utf8]{inputenc}
\usepackage{amsmath}
\usepackage{geometry}
\definecolor{mycolour}{RGB}{46,52,64}
\pagecolor{mycolour}
\color{white}

\title{Funzioni}
\author{Eugenio Animali}
\date{14 09 2022}
\begin{document}

\maketitle
\tableofcontents
\newpage
\section{Studio del Dominio}
Dalle Condizioni di Esistenza troviamo il Dominio:
\begin{align}
    \frac{g(x)}{h(x)}&\to h(x)\neq 0\\
    \sqrt[n]{g(x)}&\to g(x)\geq 0\text{ se $n$ pari}\\
    \log_{f(x)}(g(x))&\to \begin{cases}
        f(x)>0\\
        g(x)>0
    \end{cases}\\
    &\vdots
\end{align}
E iniziamo il grafico mettendo nel grafico i punti di discontinuitá.
\begin{itemize}
    \item Riga verticale trattegiata per discontinuitá
    \item zone oscurate per zone escluse dal dominio
\end{itemize}
\subsection{Simmetria}
Se il Dominio é simmetrico, la funzione potrebbe essere Pari o Dispari. Studio $f(-x)$:
\begin{gather*}
    =f(x)\to\text{ Pari}\\
    =-f(x)\to\text{ Dispari}
\end{gather*}
\section{Intersezione con Assi}
Per definire il segno della funzione, dobbiamo definire delle zone dove la funzione rimane o sopra o sotto all'asse x, trovando i punti di confine tra queste zone. Una funzione puó cambiare segno solo in due casi:
\begin{enumerate}
    \item Funzione definita a tratti $\to$ basta riprendere i confini definiti dalla funzione stessa
    \item passando per $I_x\to y=0$
\end{enumerate}
Tanto vale studiare anche $I_y\to x=0$.
\section{Studio del Segno}
Ora possiamo studiare il segno. Ponendo $f(x)\geq 0$, troviamo la gamma di valori $x$ dove la funzione é positiva o uguale a zero, e possiamo oscurare in quella gamma tutta la zona sotto l'asse $x$, perché sappiamo che $f(x)$ non é negativa lí. Allo stesso modo possiamo oscurare la zona positiva, lá dove $f(x)$ é negativa.
\section{Comportamento agli Estremi}
Vediamo come si comporta $f(x)$ in ogni confine nel Dominio. Per confini definiti (dove il dominio presenta una parentesi quadra), basta sostituire il valore di x per trovare il punto di confine. Per confini non definiti (dove il Dominio presenta una parentesi tonda) e per gli estremi infiniti, dobbiamo fare il limite per $x$.

\subsection{Continuitá}
\begin{gather*}
    y=f(x)\text{ é continua in }x_0\in D\text{ se }\lim_{x\to x_0}f(x)=f(x_0)
\end{gather*}

\subsubsection{3 teoremi sulle funzioni continue}

\begin{enumerate}
    \item Teorema di Weierstrass
    
    Se f é una funzione limitata e chiusa, esitono punti massimo e minimo.
    \item Teorema dell'esistenza degli zeri
    
    Se ho una funzione che ha i due estremi di segno opposto, la funzione passa per l'asse x
    \item Teorema dei valori medi
    
    Una fuzione chiusa e limitata assume almeno una volta tutti i valori intermedi tra il massimo e il minimo.
\end{enumerate}
\subsection{Discontinuitá}
Quando si distacca la funzione, parliamo di discontinuitá. Vi sono 3 specie da differenziare seguendo queste regole:
\begin{enumerate}
    \item la funzione esiste ma salta verticalmente. - prima specie
    \item la funzione tende ad infinito da una o due parti. - seconda specie
    \item c'é un buco nella funzione. basta dare un valore al buco per eliminare la discontinuitá. - Eliminabile
\end{enumerate}
\subsection{Forme Indeterminate}
Le Forme Indeterminate sono 7: $[+\infty - \infty, \frac{\infty}{\infty}, \frac{0}{0}, \infty\cdot 0, 0^0, \infty^0, 1^\infty]$

Si risolvono in diversi modi:
\begin{itemize}
    \item Per i polinomi con $x$ che tende ad infinito, si estrae la massima potenza di $x$, e tutte le altre potenze di $x$ tenderanno a zero.
    \item Per $\frac{0}{0}$, bisogna estrarre un fattore che sta nel numeratore e nel denominatore.
    \item Per $\frac{\infty}{\infty}$, si usa la legge di De L'Hopital per cui:
    \begin{gather*}
        \lim_{x\to\infty}\frac{f(x)}{g(x)}=\lim_{x\to\infty}\frac{f'(x)}{g'(x)}
    \end{gather*}
\end{itemize}
\subsection{Infiniti e Infinitesimi}
$f(x)$ é infinito per $x\to x_0$ se $\lim_{x\to x_0} f(x)=\infty$

$f(x)$ é infinitesimo per $x\to x_0$ se $\lim_{x\to x_0} f(x)=0$
\subsubsection{Gerarchia degli Infiniti}
per $x\to+\infty$ e $a>0$:
\begin{gather*}
    log_ax<x^\alpha<a^x
\end{gather*}
\subsubsection{Ordine di Infinitesimi}
Studio $\lim_{x\to x_0}\frac{f(x)}{g(x)}$

Se viene $0$, $f(x)$ arriva prima.

Se viene $n$, hanno lo stesso ordine.

Se viene $\infty$, $g(x)$ arriva prima e ha ordine superiore.
\subsection{Limiti Notevoli}
\begin{enumerate}
    \item $\lim_{x\to0}\frac{\sin x}{x}=1$
    \begin{gather*}
        \sin x<x<\tan x\\
        1<\frac{x}{\sin x}<\frac{1}{\cos x}\\
        \cos x<\frac{\sin x}{x}<1\\
        \text{per $a<b<c$, se $a$ e $c$ tendono allo stesso numero,}\\
        \text{anche $b$ tenderá allo stesso numero.}
    \end{gather*}
    \item $\lim_{x\to 0}\frac{1-\cos x}{x}=0$
    \begin{gather*}
        \lim_{x\to 0}\frac{(1-\cos x)(1+\cos x)}{x(1+\cos x)}\\
        \lim_{x\to 0}\frac{\sin^2x}{x(1+\cos x)}\\
        \lim_{x\to 0}\frac{\sin x}{x}\cdot\frac{\sin x}{1+\cos x}
    \end{gather*}
    \item $\lim_{x\to 0}\frac{1-\cos x}{x^2}=\frac{1}{2}$
    \begin{gather*}
        \lim_{x\to 0}\frac{(1-\cos x)(1+\cos x)}{x^2(1+\cos x)}\\
        \lim_{x\to 0}\frac{\sin^2x}{x^2(1+\cos x)}\\
        \lim_{x\to 0}\frac{\sin^2 x}{x^2}\frac{1}{1+\cos x}
    \end{gather*}
\item $\lim_{x\to\infty}\left(1+\frac{1}{x}\right)^x=e$
\end{enumerate}
\subsection{Asintoti}
\subsubsection{Asintoto Orizzontale}
Cerchiamo un asintoto orizzontale.
\begin{gather*}
    y=\lim_{x\to\infty}f(x)
\end{gather*}
\subsubsection{Asintoto Obliquo}
Se non c'é orizzontale, possiamo cercare uno obliquo.
\begin{gather*}
    m= \lim_{x\to +\infty}\frac{f(x)}{x}\\
    q=\lim_{x \to +\infty}[f(x)-mx]
\end{gather*}
\section{Studio Della Derivata Prima}
\subsection{Punti di Non Derivabilitá}
Bisogna studiare punti di non derivabilitá solo dove $f$ é continua, perché non ha senso studiare la derivabilitá dove $f$ non esiste. Studiamo i limiti destro e sinistro di ogni confine incluso nella definizione a tratti della derivata:
\begin{enumerate}
    \item Flesso a Tg Verticale\\
    la tangente verticale non ha coefficiente angolare. I limiti destro e sinistro della derivata vanno a $\pm$ infinito
    \item Cuspide\\
    spigolo di curve per cui i limiti destro e sinistro vanno uno a + infinito e uno a - infinito
    \item Punto Angoloso\\
    limiti destro e sinistro sono diversi e almeno uno non é infinito. si chiama cosí perché ha un angolo
\end{enumerate}
Per trovare di quale si tratta facciamo i limiti sinistro e destro della derivata.
\subsection{Andamento}
\begin{gather*}
    \text{studio }f'(x)\geq 0\text{ per trovare:}\\
    f'(x)>0\to \text{ Funzione Crescente}\\
    f'(x)<0\to \text{ Funzione Decrescente}\\
    f'(x)=0\to\text{ Punto Stazionario}
\end{gather*}
\begin{enumerate}
    \item Cresce, Punto Stazionario, Decresce$\to$ Massimo Relativo ($M$)
    \item Decresce, Punto Stazionario, Cresce$\to$ Minimo Relativo ($m$)
    \item Decresce, Punto Stazionario, Decresce$\to$ Flesso Orizzontale
    \item Cresce, Punto Stazionario, Cresce$\to$ Flesso Orizzontale
\end{enumerate}
\section{Studio della Derivata Seconda}
\begin{gather*}
    f''(x)>0\to\text{Concavitá verso l'Alto}\\
    f''(x)<0\to\text{Concavitá verso il Basso}\\
    f''(x)=0\to\text{Flesso (Cambio di Concavitá)}
\end{gather*}
\begin{gather*}
    \lim_{x\to\infty}\frac{x}{\sin x}=1
\end{gather*}
\end{document}