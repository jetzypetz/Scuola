\documentclass{article}

\usepackage{tikz}
\usepackage{parskip}
\usepackage{amssymb}
\usepackage[utf8]{inputenc}
\usepackage{amsmath, empheq}
\usepackage[margin=3.75cm]{geometry}
\definecolor{mycolour}{RGB}{46,52,64}
\pagecolor{mycolour}
\color{white}

\title{\jobname}
\author{Eugenio Animali}

\begin{document}
\maketitle

\section*{Introduzione}

\begin{itemize}

\item RIFIUTO ANTICO REGIME: Barocco, periodo di sfoggio ecnclesiale di potere.$\to$ Illuminismo, che valorizza la ragione; tema che riporta all'umanesimo(di periodo Rinascimentale), che riproposto.
\item Teorico tedesco Winkelmann $\to$ bellezza ideale; sviluppata anche da canova- Kalokagathia
\begin{center}
    Bellezza ideale- inespressiva perché rappresenta l'interiore e i sentimenti esteriori mascherano l'interioritá
\end{center}
\item Scoperte Pompei ed Ercolano - torna di moda $\to$ Si diffonde il grand tour- Viaggiare a vedere storia antica, Rinascimentale.
\item Rivoluzione Industriale
\end{itemize}
\section*{Cosa succede}
\begin{itemize}
    \item acquerello
    \item architettura - si riprende il tempio Greco, ordine dorico.
\end{itemize}
\section*{Canova}
\begin{itemize}
    \item Artista che incarna al 100\% tutti gli ideali di Winkelmann
    \item Grande successo
    \item Pensavano che le sculture greche non fossero pitturate- sculture solo bianche
    \item Tutti volti inespressivi
    \item Bravissimo a disegnare, disegna prima ed é facilmente comprensibile la scultura nel disegno, modella in creta, poi in cartongesso, poi faceva scolpire ai suoi alunni, lasciando il rifinimento a Canova, poi applica una cera rosa per avvicinare al colore dell'incarnato
\end{itemize}
\begin{enumerate}
    \item Teseo sul Minotauro
    \begin{itemize}
        \item Mitologico
        \item Non bisogna rappresentare il pieno dell'azione - la lotta é giá successa, vediamo il momento dopo, perché non vogliamo vedere il pathos.
        \item Struttura piramidale
    \end{itemize}
    \item Amore e Psiche
    \begin{itemize}
        \item Psiche apre una scatola datale da Venere che la fa addormentare per tanto tempo; la sorregge Amore
        \item Non si toccano molto, non atto di passione; giá svegliata
        \item Visione a $360\deg$- technica Barocca; da diversi angoli si vedono cose diverse.
        \item Linee compositive incrociate (intreccio tra personaggi)
    \end{itemize}
    \item Amore e Psiche Stanti
    \begin{itemize}
        \item Piú giovani - piú perfezione
        \item Psiche da ad amore una farfalla - simbolo di purezza
        \item Superfice delicata
    \end{itemize}
    \item Venere Vincitrice
    \begin{itemize}
        \item Sorella di Napoleone, Paolina Bonaparte, trasformata in divinitá (volto idealizzato)
        \item Scultura ruota dal lettino
    \end{itemize}
    \item Ebe
    \begin{itemize}
        \item Contrasto tra parti nuda e coperta
        \item Poggiata su una nuvola
        \item Movimento del vestito distacca dal Neoclassicismo, molto piú vicino al Barocco
    \end{itemize}
\end{enumerate}
\section{Jacques Louis David}
Parigino, pittore

come canova, molto fissato all'idea del disegno.

studi del corpo
\subsection*{Vita}
\begin{itemize}
    \item Nasce a parigi 1748
    \item accademia delle belle arti di parigi
    \item Roma napoli, ercolano, Pompei
    \item Partecipa alla rivoluzione Francese- amato da Napoleone.
\end{itemize}
Prix de Rome 1775- Premio della accademia di parigi a Roma
\subsection{Orazi e Curiazi}
Scelta di Episodio classico - per comunicare con i suoi contemporanei neoclassici e rivoluzionari.

Donna in basso disegnate con linee morbide per accentuare ancora di piú la forza ed il valore di questi uomini, che combattono per la loro patria.
\subsection{La Morte di Marat}
Qualitá neoclassica di non rappresentare l'azione, ma il prima o il dopo; senza riempire la scena di cose per suggerire l'avvenuto.
\subsection{Le Sabine}
Anche se v'é tanto movimento, Ersilia in mezzo blocca tutto. Figlia di Tazio e Moglie di Romolo.

Attenzione particolare in rispetto alla muscolatura. Per ottenere questa accentuazione, vi é una alternanza di verso, per la quale alcuni personaggi mostrano il dorso.
\subsection{Napoleone che va a San Bernardo}
Tutto bellissimo, irrealisticamente sembra un modello. Non vediamo la battaglia, ne i morti.
\section{2 Filoni}
Da qui seguono due filoni artistici che seguono uno dei due:
\begin{enumerate}
    \item Neoclassicismo - copia della realtá, perfezionamento
    \item Romanticismo - Fantasia, Invenzione, Sentimento
\end{enumerate}
\section{Questo quadro mi suscita...}
Pagliai di monet, constable
\end{document}