\documentclass{article}

\usepackage{tikz}
\usepackage{parskip}
\usepackage{amssymb}
\usepackage[utf8]{inputenc}
\usepackage{amsmath, empheq}
\usepackage[margin=3.75cm]{geometry}
\definecolor{mycolour}{RGB}{46,52,64}
\pagecolor{mycolour}
\color{white}

\title{\jobname}
\author{Eugenio Animali}

\begin{document}
\maketitle


\section{Periodo ipotetico}
\begin{enumerate}
\item se + Ipotesi $\to$ Protasi condizionale, Oggettivitá e Realtá
\subsection{Latino}

\subsubsection{Protasi}
si/nisi + Indicativo
\subsubsection{Apodosi}
Indicativo

\end{enumerate}
\subsection{Realtá o Oggettivitá}
\subsubsection{Protasi}
\subsubsection{Apodosi}
\subsection{Possibilitá}
\subsubsection{Protasi}
Se i nemici attaccassero (it: se + congiuntivo presente; lat: si/nisi + congiuntivo o presente o perfetto)
\subsubsection{Apodosi}
io scapperei (it: condizionale presente; lat: congiuntivo presente)
\subsection{Irrealtá}
\subsubsection{Protasi}
se avessimo speso bene la vita (it: se + congiuntivo piúccheperfetto; lat: si/nisi + congiuntivo imperfetto)
\subsubsection{Apodosi}
avremmo mangiato piú porchetta (it: condizionale piúccheperfetto; lat: congiuntivo imperfetto o piúccheperfetto)
\end{document}