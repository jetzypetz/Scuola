\documentclass{article}

\usepackage{tikz}
\usepackage{parskip}
\usepackage{amssymb}
\usepackage[utf8]{inputenc}
\usepackage{amsmath}
\usepackage[margin=3.75cm]{geometry}
\definecolor{mycolour}{RGB}{46,52,64}
\pagecolor{mycolour}
\color{white}

\title{\jobname}
\author{Eugenio Animali}

\begin{document}
\maketitle

\section{Storia}
Il Regno d'Italia nasce nel 1861 come monarchia costituzionale sotto la dinastia dei Savoia. Viene unificato grazie ad una guerra da parte del Regno Sabaudo che vince man mano contro ogni regione italiana, fino ad annetterle tutte apparte lo Stato Pontificio e il Nordest, che rimane Austriaco. Al resto dell'Italia viene sottratta tutta l'autonomia, e venne estesa la legislazione sabauda, che era in vigore in Piemonte, per quanto riguarda l'amministrazione, l'apparato fiscale (Tasse sugli alimenti), la scuola (2 anni obligatori di scuola elementare), l'esercito (5 anni di leva militare).
\subsection{Destra Storica (1861-1876)}
La Destra Storica é il primo partito ad andare al potere, e attua una politica agricola. L'Italia é arrivata tardi all'industrializzazione, e la maggior parte della sua economia é ancora agraria. Il voto censitario significa che votano solo i signorotti (proprietari terrieri) che costituiscono il 2\% della popolazione italiana; costoro non possono che votare una politica che favorisca i commerci, per ridurre i propri costi di esportazione (tasse doganali basse). Ne vale una situazione Italiana fortemente arretrata, in mano ai grandi proprietari terrieri, lavorazione estensiva, senza macchine, senza modernizzazione.
\subsection{La Sinistra}
Protezionismo 
\section{Contesto}
Italia nasce come Monarchia con i Savoia. Con una guerra si prendono tutti regni man mano annettendole al regno sabbaudo. Piemontesizzazione generale per cui tutta la cultura politica e sociale si estendette dal Piemonte. La scuola, l'esercito, le leggi sono estesi direttamente a tutta italia (primo problema perché é vista come annesione) apparte Roma e Nordest che rimane austriaco. Primo partito che va al potere é la destra storica 61-76 che impone politica di sviluppo agricolo e commerciale d'Italia; poiché l'Italia sta indietro con l'industrializzazione. É un ritorno ad una agricoltura arretrata, poiché . Sistema di voto é censitario (2\% dell'italia vota) significa che solo i signorotti proprietari terrieri possono votare.
\begin{center}
Cinque anni di leva militare. Tasse sugli alimenti rendono piú debole una situazione gia precaria.

Si parte da un alfabetismo in Italia del 98\%. Savoia impongono Scuola obbligatoria (2 anni).
\end{center}
Alla destra succede la sinistra, che aumenta i dazii allentrata in Italia, quindi aumentando lo sviluppo dell'economia in Italia.

Inizia una guerra di dazii tra paesi che danneggia l'italia.
\begin{center}
Problema della lingua: quale lingua usare?

Risponde Manzoni, ma é stata lunga la conversione. Aiutó la migrazione interna e la leva militare, che faceva parlare persone di diverse zone.
\end{center}
\subsection{Positivismo}
Ottimismo di poter utilizzare la ragione per arrivare a controllare tutto della vita umana per creare la vita perfetta. Il metodo della conoscenza va esteso a tutti i campi. Per questo abbiamo lo studio della psicologia.
\section{Scapigliatura}
Industrializzazione $\to$ svalutazione degli artisti

Bohemiens parigini $\to$ si pensa che gli zingari vengano dalla boemia. Massimo esponente dei Bohemienne é Baudelaire.

Scapigliati sono gli maledettisti che non vogliono conformare al perbenismo. Gruppo di Intellettuali giovani a Milano appena dopo l'unitá d'Italia. (Non in Sicilia perché non c'é industrializzazione). Parlano nelle loro poesia della modernitá che li circonda; ma la disprezzano e rifiutano nelle loro descrizioni. Peró non sono idealisti e sanno che non si potrá tornare ad un mondo preindustrializzato. Melancolia. Dualismo (rifiuto e accettazione).

Nome inventato da Cletto Arrighi che lo pone in una sua opera.

\subsection{Arrigo Boito}
\subsubsection{Vita}
Nato a Padova, formazione musicale e borghese (anticonformismo nasce nel seno della borghesia stessa), diventa senatore.
\begin{center}
Case nuove: non é all'avanguardia nella forma. nella tematica si.

Strumenti di distruzioni- di strage.

``l'impero é vostro'' accettazione

``O tempi irrequieti'' Melancolia

``ratta'' il progresso veloce

``piangan pure i poeti'' ogni tanto ha momenti di caduta nel dolore tra le false invocazioni della modernitá

Gli augelletti e i ciechi non ritrovano la strada come il poeta. In questa societá non hanno spazio gli artisti
\end{center}
\section{Naturalismo francese (1870)}
\subsection{Hyppolite Taine}
Questo movimento ha come precedente il positivismo(abbandono di ogni metafisica perché la scienza puó spiegare tutto, e renderá migliore la vita degli uomini; espressione ideologica della nuova organizzazione industriale). Si basa sulle idee di Hippolyte Taine. Taine segue un rigido materialismo e determinismo, per cui i sentimenti sono solo prodotto della fisiologia umana e sono determinati e determinabili dall'ambiente fisico in cui vive il soggetto. Taine riconosce nel romanzo una natura analitica e critica che lo avvicina alla scienza in quanto ``grande inchiesta sull'uomo''. Per ció il romanziere deve essere un romanziere scienziato. Le opere letterarie devono indagare la vita degli uomini come sperimenti mentali.
\subsection{I Precursori}
\subsubsection{Honoré de Balzac}
Taine si rifa a Honoré de Balzac come suo modello per il quadro critico della societá francese nell'etá della Restaurazione che é la \emph{Commedia Umana}.
\subsubsection{Gustave Flaubert}
Taine si interessa della teoria dell'impersonalitá di Flaubert. ``L'artista deve essere nella sua opera come Dio nella creazione, invisibile e onnipotente, sí che lo si senta ovunque, ma non lo si veda mai. E poi l'Arte deve innalzarsi al di sopra dei sentimenti personali e delle suscettibilitá nervose.''
\subsection{Emile Zola}
Nel \emph{Romanzo Sperimentale} (1880) Zola si pone come figura principale del Naturalismo per la sua dedicazione a svilupparne le idee. Secondo Zola, la scienza, che é stata applicata prima ai corpi inanimati (chimica, fisica), poi ai corpi animati (biologia), ora va applicata alle sensazioni ``spirituali'' (pensiero, passione $\to$ psicologia). Segue che la letteratura, che si occupa di queste, deve far parte della scienza (Romanzo che analizza la realtá psicologica $\to$ Romanzo Sperimentale).

Scrisse famosa lettera ``J'accuse'' sul scandaloso affare Dreyfus, capitano francese accusato a torto di tradimento e condannato a deportazione. Zola accusa il sistema giuridico di antisemitismo, per aver condannato senza prove Dreyfus. Zola viene condannato anche lui ad un anno di carcere ed una multa.
\subsubsection{Opere importanti}
Scrive 20 romanzi per analizzare una famiglia e descrivere i loro comportamenti. Scrive anche La teoria per descrivere ció che intende fare con queste opere: rendere la letteratura una scienza.
\subsubsection{Il Modello}
Flaubert é il modello per Zola. ne prende la teoria della impersonalitá. No all'onniscienza (Manzoni che ci dice gli antefatti, i pensieri e i flashback, con commenti morali); Secondo Flaubert, l'artista deve essere nella scrittura come Dio nella creazione: evidentemente c'é ma non lo si deve vedere. ``Il narratore deve essere il Dio nascosto del racconto."
\subsection{Flaubert}
\subsubsection{Vita}
Flaubert: Padre é un chirurgo. In Madame Bovary, il marito é medico chirurgo.
\subsubsection{Storia}
Emma Bovary ha letto tanti romanzi, che le hanno creato una immagine ideale dell'amore. Quindi si innamora con Charles, pensando che sarebbe stata una storia romanzesca, ma si rende presto conto che non é innamorata di lui, ma dell'idea di amore che aveva letto. Lei si ammala per questo pensiero, e il marito decide che si devono spostare in campagna. Lei é preda di un dongiovanni, Rodolphe, ricco proprietario della zona.
\subsubsection{Modo in cui é scritta}
Non c'é mai un commento esplicito. Pensiero scritto come discorso indiretto libero: non vengono introdotti i pensieri, ma messi direttamente, come se si vedessero direttamente, perché non vi é una persona dello scrittore da far da mediatore: impersonalitá.
\section{Verismo}
Giovanni Verga, Federico de Roberto, Capuana

Scrittori italiani (sempre Milano. Piú contatto con ció che succede in europa) che si rifanno al Naturalismo francese. Capuana accetta la impersonalitá di Zola e la teorizza su Il Corriere della Sera, ma non é d'accordo con la scientificitá della letteratura. (``Senza dubbio l'elemento scientifico s'infiltra nel romanzo contemporaneo; ma la vera novitá non istá in questo. Né stá nella pretesa di un romanzo sperimentale, bandiera che lo Zola inalbera arditamente, a sonori colpi di grancassa. Un'opera d'arte non puó assimilarsi un concetto scientifico che alla propria maniera, secondo la sua natura d'opera d'arte.''). Invece Verga ne scrive i romanzi.

\subsection{Rosso Malpelo}

\section{Domande}
Dovevamo fare inizio pg.99?

\end{document}