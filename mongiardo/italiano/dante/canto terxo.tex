\documentclass{article}

\usepackage{parskip}
\usepackage[utf8]{inputenc}
\usepackage{amsmath, empheq}
\usepackage[dvipsnames]{xcolor}
\usepackage[margin=3.75cm]{geometry}
\definecolor{mycolour}{RGB}{46,52,64}
\pagecolor{mycolour}
\color{white}

\title{Canto 3}
\author{Eugenio Animali}
\date{16 09 2022}

\begin{document}

\maketitle

\section*{La luna - Astro degli incostanti}
\section{tema}
Pur essendo poste in cieli diversi, le anime non sono invidiose. Adeguano la loro volontá alla volontá di dio (sono stati messi la e sono contenti).
\section*{Storia}
Vede una luce, non insopportabilmente brillante, perché trova anime meno beate. Quindi si vede un poco la fattezza umana, come quando ci si rispecchia nell'acqua. Vedendole sfuocate, pensa siano riflessi e si gira per vedere la persona vera; primo sbaglio di Dante, opposto a quello di Narciso. I beati ridono, e lo rassicurano che si puó fidare. Dante chiede alla prima chi é. É Piccarda che era amica di Dante ma non la riconosce, perché é piú bella in paradiso. Piccarda spiega la sua caritá, e amore per la posizione datale. Dante si scusa per non averla riconosciuta prima. Viene spiegato a dante come non c'e desiderio. Dante chiede perché non ha portato a termine il voto (metafora tessile). (redicenza) inizia con santa chiara, che creó l'ordine, spiega a grandi linee, senza usare nomi; anche dopo, in cuore ha mantenuto sempre il velo. Dante scrive una leggenda per screditare Federico II
\section{Note}
Caritá- non é dare soldi al povero, ma l'amore per dio. Amando Dio (avendo Caritá), vogliono ció che vuole dio. Nell'amore, si vuole ció che vuole la persona amata.
\section*{Persone}
Piccarda Donati rapita dal convento dal Fratello, per farla sposare ad un capo politico.

Costanza d'Altavilla Madre di Federico II
\section{parole}
latino - facile
vento di soave - Imperatore Enrico 6 di Svevia (Federico I Barbarossa $\to$ Enrico VI + Costanza d'Altavilla $\to$ Federico II)
incela - mette in cielo
capere - essere
\end{document}