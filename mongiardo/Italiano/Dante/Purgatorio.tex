\documentclass{article}
\usepackage[utf8]{inputenc}
\usepackage{amsmath, empheq}
\usepackage[dvipsnames]{xcolor}
\usepackage[margin=3.75cm]{geometry}
\definecolor{mycolour}{RGB}{46,52,64}
\pagecolor{mycolour}
\color{white}

\title{Paradiso}
\author{Eugenio Animali}
\date{13 Settembre 2022}

\begin{document}

\maketitle

\section*{Cosmologia Dantesca}

Guardare libro, pagine di Introduzione.
Due emisferi, tra Gange e colonne d'Ercole;

\section*{Purgatorio, Spunti per Ricordare}
natural burella; spiagge del purgatorio, con Catone il Censore; Stazio; Matelda; Beatrice, pianto e bere dal Leté, bagno nel Leonoé; "se avessi piu spazio da scrivere, descriverei com'era la mia purificazione"

\section*{Paradiso}
Niente paesaggio terrestre, solo luce.\\
Sfera del Fuoco:\\
Tra Purgatorio e Paradiso, vi é la Sfera del Fuoco che attira il fuoco dalla terra, motivo per cui tende in sú.\\
\\
9 Cieli:\\
Cerchi concentrici che girano, piú il 10 Cielo, Empireo, statico.
Ciascuno prende il nome dell'astro importante incastonato dentro.\\
Sono spostati da una musica celeste, che non si sente dalla terra, ma lo sente Dante.\\
Qui Dante incontra i beati, che gli vengono incontro mandati da Dio. Ciascun beato va al pianeta che lo ha influenzato in vita, come descritto da Marco Lombardo nell'Inferno. (I pianeti iniziano i nostri movimenti, poi abbiamo il libero arbitrio).\\
\\
Empireo:\\
Sede di Dio, circondata da nove cerchi angelici. Gli angeli sono luce\\
\\
Candida Rosa:\\
Un anfiteatro dove stanno i beati. Questi contemplano Dio e cosí si beano. Vi sono diversi gradi di beatitudine, ma non c'é invidia. Anch'essi sono luce, piú o meno brillante.
\end{document}