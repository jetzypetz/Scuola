\documentclass{report}

\usepackage{tikz}
\usepackage{parskip}
\usepackage{amssymb}
\usepackage[utf8]{inputenc}
\usepackage{amsmath, empheq}
\usepackage[margin=3.75cm]{geometry}
\definecolor{mycolour}{RGB}{46,52,64}
\pagecolor{mycolour}
\color{white}

\title{\jobname}
\author{Eugenio Animali}

\begin{document}
\maketitle
\tableofcontents

\chapter{Teoria}

\section{Vita}
Nato 29 giugno 1798 a Recanati.
\subsection*{Infanzia Erudita 1798}
Famiglia decadente, con padre reazionario che si dedica solo ai sui libri attardati e madre autoritaria e fredda. In una condizione di solitudine, Leopardi si dedica da piccolo al'erudizione. Avendo imparato tutto ció che il precettore poteva insegnargli, si dedica a leggere tutta la biblioteca del padre per "sette anni di studio matto e disperatissimo", arrivando a delle opere erudite di traduzioni e storiografia giá a 13 anni. Comunque in questo periodo é inevitabilmente limitato all'esposizione a informazioni antiquate e illuministiche.
\subsection*{Dall'erudizione al Bello 1816}
Abbandono degli studi eruditi e si dedica ai grandi poeti del passato e gli scrittori moderni del romanticismo (De Stael) grazie all'amicizia con Pietro Giordani, intellettuale classicista.
\subsection*{Dal Bello al Vero 1819}
Tentativo di fuga fallisce, malattia agli occhi $\to$ Periodo di abbandono al pensiero che porta ad uno studio accurato di filosofia. Quindi un continuo cambiare interesse di studio, culminante con l'Infinito.
\subsection*{Prima Uscita da Recanati 1822}
Opportunitá di uscire a roma dallo zio. Delusione perché non trova movimento intellettuale che sperava di trovare. Tornato a casa, si dedica alle \emph{Operette Morali} che eprimono il suo pensiero pessimistico $\to$ scritte in prosa a causa di un'ariditá intellettuale derivata dalla delusione.
\subsection*{Autonomia economica 1825}
Lavora per l'editore Stella per vari lavori (traduce Cicerone, commenta Petrarca, \dots) Milano - Bologna - Firenze (in contatto con Gian Pietro Vieussaux e gruppo di intellettuali a capo di una rivista "L'Antologia"). Inverno a Pisa 1827 migliorano le condizioni di saluti e con esse ritorna la facoltá immaginativa, segno dell'inizio del \emph{risorgimento}. Nasce 1828 "A Silvia", il primo dei \emph{Grandi Idilli}.
\subsection*{1828}
Problemi economici legati a problemi di saluti portano alla rassegnazione di tornare a casa per "sedici mesi di notte orribile", chiuso in casa senza rapporti con l'esterno. Finalmente nel 1830 accetta un'offerta da parte degli amici fiorentini. Fecondo periodo di dibattito, quindi una nuova apertura a conoscere l'ambiente intellettuale esterno. Con un nuovo amico Antonio Ranieri va a Napoli in un ambiente illuministico e neo-classico, avverso a lui.

Morto 1837 a Napoli.
\section{Le Lettere}
Leopardi scrisse molte lettere straordinarie, che, pur non essendo scritte per scopo letterario di pubblicazione, come molti letterati italiani, ci offrono una finestra per vedere le la faccia piú privata e sentimentale di Leopardi, dove sfoga i suoi tormenti e le sure sofferenze.
\subsection*{Pietro Giordani}
Dalle prime lettere del 1817 a Giordani, Leopardi trova finalmente uno sfogo dalla prigionia arida di Recanati. Vede in Giordani un sostituto ddella figura paterna, che puó conversare con lui per nutrire il suo pensiero filosofico, e condividere e compredere le sofferenze di Leopardi.
\subsection*{Famiglia}
\subsubsection{Fratello Carlo}
Il rapporto con il fratello Carlo é caratterizzato da toni ironici e scherzosi.
\subsubsection{Sorella Paolina}
Con la sorella Paolina invece, Leopardi puó confidare piú intimamente, come fa in diverse lettere nel 1828: ``Ho qui in Pisa una certa strada deliziosa, che io chiamo \emph{Via delle Rimembranze}: lá vo a passeggiare quando voglio sognare a occhi aperti. Vi assicuro che in materia di immaginazioni mi pare di essere tornato al mio buon tempo antico", "Dopo due anni, ho fatto dei versi quest'Aprile; ma versi veramente all'antica e con quel mio cuore d'una volta". Qui, per esempio, si apre con Paolina sull'arrivo del periodo di risorgimento, donde ritornerá quella antica facoltá di immaginare che gli permette di ritornare ai vecchi temi nel "A Silvia".
\subsubsection{Padre Monaldo}
Rimane perenne la freddezza e lontananza del padre passatista e rigido, nonostante qualunque tentativo di affetto da parte del figlio.
\section{Il Pensiero}
\subsection*{Pessimismo Storico}
Sviluppato nel Luglio del 1820 é il primo studio sul pensiero che emerge dallo Zibaldone. Qui Leopardi fa una importante considerazione filosofica sul desiderio e la felicitá degli uomini: Gli uomini cercano la felicitá, che é un concetto infinito e puro; tant'é che un singolo momento felice non sazierá mai  questo desiderio in un uomo, in quanto rimane sempre piú felicitá a cui aspirare. Segue necessariamente che la ricerca della felicitá (interesse e ossessione dei moderni ``filosofastri") é vana e fallimentare, ed é il motivo per cui gli uomini saranno sempre legati ad una condizione nulla, di inutilitá.

Vi é ancora, peró, in questo primo tempo una speranza di nascodersi da questa condizione pessima. La natura é considerata come una provvida ``madre benigna" che offre alle sue creature un rimedio: La Immaginazione e le Illusioni aiutano a sopportare la nullitá perché nascondono questo fatto, lasciando spazio ad un mondo immaginato e quindi surreale e perfetto. Da qui segue la considerazione che gli antichi, Greci o Romani che siano, stavano meglio perché, non tanto avanzati nello sviluppo della ragione, erano piú vicini alle loro radici naturali, basandosi piú su Immaginazioni- che comportavano una forte spinta all'eroismo e la magnanimitá (estranea ad un freddo calcolo egoistico e ragionato). Invece nella modernitá vi é stata una spinta violenta verso la ragione, che ci ha portati a svelare questa nostra condizione, e dover accettare la nullitá che é la nostra esistenza; Leopardi si sente l'unico sostenitore della grande antica virtuositá, presentandosi come un Titano.
\subsection*{Pessimismo Cosmico}
Verso il 1824-1825, con una riconsiderazione della natura, Leopardi trova che invece non é interessata al mantenimento della creatura individuale, ma ansi al mantenimento della specie. Quindi per Leopardi la natura diventa un freddo processo meccanico, non piú fonte di salvezza dalla nullitá. Ansi, Leopardi pone la colpa di questa infelicitá direttamente sulla natura, che é la causa iniziale di quella propensitá crudele verso la felicitá (A questa colpa aggiunge la colpa delle malattie, i terremoti, la morte, \dots).

Mentre come considerazione filosofica, la natura é concepita come meccanismo indifferente, in poesia, e in rappresentazione artistica (come nel \emph{Dialogo tra la Natura ed un Islandese}), incorpora quell'immagine intenzionalmente malvagia e attiva. Ora non vi é piú differenza tra passato e presente, perché la natura ci circonda sempre, ed il modello di Leopardi cambia dall'eroe antico, che si nutre di quelle facolta Immaginative della natura, al saggio, che sa distaccarsi dalle noie afflitte dalla natura.

In questo secondo tempo, la felicitá infinita si puó raggiungere dall'immaginazione, che essendo libera dai vincoli della realtá, puó raggiungere qualunque livello di perfezione.``Il piacere infinito che non si puó trovare nella realtá, si trova cosí nella immaginazione, dalla quale derivano la speranza, le illusioni". Seguono nello Zibaldone diversi esempi di esperienze \emph{vaghe} che avviano quella facoltá immaginativa.
\subsection*{Citazioni}
``Sono cosí stordito dal niente che mi circonda'' \emph{Lettera a Pietro Giordani 1819}

``Sono cosí spaventato dalla vanitá di tutte le cose, e della condizione degli uomini, morte tutte le passioni'' \emph{Lettera a Pietro Giordani 1819}

``questa é la miserabile condizione dell'uomo, e il barbaro insegnamento della ragione, che i piaceri e i dolori umani essendo meri inganni, quel travaglio che deriva dalla certezza della nullitá delle cose, sia sempre e solamente giusto e vero.'' \emph{Lettera a Pietro Giordani 1820}

``Queste considerazioni io vorrei che facessero arrossire quei poveri filosofastri che si consolano dello smisurato accrescimento della ragione, e pensano che la felicitá umana sia riposta nella cognizione del vero, quando non c'é altro vero che il nulla, e questo pensiero, ed averlo continuamente nell'animo, come la ragion vorrebbe, ci dee condurre necessariamente e dirittamente a quella disposizione che ho detto la quale sarebbe pazzia secondo la natura, e saviezza assoluta e perfetta secondo la ragione'' \emph{Lettera a Pietro Giordani 1820}

``Il piacere infinito che non si puó trovare nella realtá, si trova cosí nella immaginazione, dalla quale derivano la speranza, le illusioni" \emph{Zibaldone pg.167}

``Quello che ho detto altrove degli effetti della luce, del suono, e d'altre tali sensazioni circa l'idea dell'infinito, si deve intendere non solo di tali sensazioni nel naturale, ma nelle loro imitazioni ancora, fatte dalla pittura, dalla musica, dalla poesia etc. Il bello delle quali arti, in grandissima parte [\dots] consiste nella scelta di tali o somiglianti sensazioni indefinite da imitare" \emph{Zibaldone pg. 1983}

``Vi assicuro che in materia di immaginazioni mi pare di essere tornato al mio buon tempo antico" \emph{Lettera a Paolina 1827}
\section{I Canti}
\subsection*{Cronologia}
I Canti é l'opera piú importante di Leopardi, che comprende la maggior parte dei suoi scritti. La prima edizione viene pubblicata a Firenze nel 1831, comprendendo alcune di quelle sperimentazioni prima del 1819, 10 \emph{Canzoni} (1818-1823), due \emph{Elegie}, sei \emph{Idilli}, e altre opere recenti. Pubblicó un'altra edizione a Napoli nel 1835, e Ranieri ne pubblicó la terza dopo la sua morte.
\subsection*{Le Canzoni (prima fase 1818-1821, seconda fase 1821-1823)}
Dieci componimenti in stile classicistico, che riprendono uno stile aulico ed uno schema metrico fissato sin dalla lirica duecentesca e Dante. Sono sublimi e seguono fortemente la tradizione. Le prime 5 affrontano la tematica civile, con la impostazione del pessimismo storico, quindi criticando l'etá presente (inerte, corrotta, e incapace di eroismo). Nel \emph{Bruto minore} e \emph{L'Ultimo Canto di Saffo}, non parla piú in prima persona, ma passa il racconto ai due protagonisti, entrambi titani suicidi (anche se storici $\to$ si noti che é finito il pessimismo storico.). Siamo nella fase intermedia, nella quale Leopardi incolpa il fato, e nella situazione antica, anche gli dei della sofferenza umana. Per questo i protagonisti si danno all'atto piú estemo di ribellione ad una vita cosí sofferente- il suicidio. \emph{Alla primavera} é una rievocazione delle ``favole antiche", che esibiscono l'immaginositá degli antichi. Anche \emph{L'Inno ai Patriarchi} é una rievocazione dei felici primitivi, mentre \emph{Alla sua donna} é dedicata ad una immagine ideale, platonica di donna.
\subsection*{Gli Idilli 1819-1821}
\emph{Gli Idilli} sono di natura molto diversa dalle \emph{Canzoni} sia per il linguaggio piú diretto e semplice, che negli argomenti, che sono intimi, autobiografici. Leopardi riprende la tradizione letteraria degli Idilli iniziata da Teocrito avendo tradotto prima del 1819 gli Idilli pastorali di un imitatore di Teocrito, Mosco (II a.C.). Peró Leopardi non scrive piú Idilli secondo quella tradizione bucholica di campagna stilizzata e figure idealizzate di pastori; Non ha neppure a che fare con la nozione moderna di idillio, quell'idillio ``borghese'' che si era affermato nel Settecento nelle letterature nordiche e che amava rappresentare scene della vita quotidiana di pesonaggi di mediocre condizione. Negli \emph{Idilli}, dunque, Leopardi propone invece i ``sentimenti, affezioni, avventure storiche del suo animo'', che sono da considerare come una rappresentazione soggettiva della realtá esterna.

\emph{L'infinito} puó ricordare molto uno scenario dell'Idillio classico (la siepe, il vento sulle foglie), ma non é lo scenario di una semplice quiete contemplativa e rasserenante, bensí lo spunto per una vertiginosa meditazione lirica sull'idea di infinito creato dall'immaginazione.

\emph{Alla Luna} affronta invece il tema complementare della ricordanza, che trasfigura il reale e l'abbelisce.

\emph{La sera del dí di festa} inizia con un notturno lunare, suggestivo e vago, per allargare ad una meditazione sul tempo che cancella ogni traccia dell'azione umana.
\subsection*{L'Arido Vero 1823- Primavera 1828}
Leopardi passa un periodo di silenzio poetico a causa di una ``fine delle illusioni giovanili''. Vuoto di ispirazione, si dedica all'investigazione dell'``arido vero'', la filosofia. In questo periodo nascono le \emph{Operette Morali}, che corrispondono al passaggio al pessimismo assoluto. Ne deriva un abbandono degli atteggiamenti titanici degli anni precedenti e una disposizione piú distaccata e ironica nei confronti della realtá.
\subsection*{Il Risorgimento e i Grandi Idilli 1828-1830}
La svolta si verifica nel periodo relativamente felice trascorso a Pisa tra Inverno e Primavera del 1828. Si vede nella lettera a Paolina questo ritorno delle ``illusioni giovanili'': ``ho fatto dei versi quest'Aprile; ma versi veramente all'antica e con quel mio cuore d'una volta''. Leopardi assiste a un ``risorgimento'' della sua ispirazione. Infatti nell'aprile del 1828 scrive \emph{Il risorgimento}, che spiega il ritornare di quel ``dolce affanno'' che rende graziose tutte le esperienze. Pochi giorni dopo nasce \emph{A Silvia}. Anche quando ritorna a Recanati alla fine dell'anno per sedici mesi di ``notte orribile'', non perde quella ispirazione che gli permette di scrivere diversi \emph{Grandi Idilli}. Il termine non é di Leopardi, ma viene usato come estenzione per il fatto che riprendono gli stessi temi suggestivi e vaghi, e lo stesso linguaggio limpido, lontano dell'aulicitá. Differiscono dai primi idilli per la maturitá acquisita da quegli studi dell'``arido vero'' di quegli anni, e lo sviluppo del pessimismo assoluto. Pur invocando bei ricordi, e immagini pittoresche, rimane in sottofondo la consapevolezza della nullitá della vita e la falsitá delle immagini stesse. La consapevolezza del vero non distrugge il loro valore; equilibrio tra il ``caro immaginare'' e il ``vero''. Si noti che non vi sono piú azioni titaniche (\emph{Sera del dí di festa} - ``qui per terra mi getto, e grido, e fremo''). Leopardi ha imparato dalle \emph{Operette Morali} una contemplazione piú ferma. Segue che anche il linguaggio sia piú contemplativo che titanico e passionale.
\subsection*{Ciclo di Aspasia 1830-1835}
Dopo il periodo delle \emph{Operette Morali}, distaccate e aride; dopo il periodo dei \emph{Grandi Idilli}, introspettivi e contemplativi; dopo La ``notte orribile''; Leopardi finalmente accetta l'offerta mensile degli amici fiorentini. A Firenze, trova un nuovo ambiente culturale e letterario (quel che aveva cercato a Roma) e Inizia un periodo piú discorsivo, con dibattiti continui con gli ottimisti liberali. Si vedrá nel \emph{Ciclo di Aspasia} un fiorire di uno stile piú combattivo, che replica la sicurezza nelle sue idee di Leopardi durante i dibattiti. In questo ambiente sociale, finalmente Leopardi puó fare una amicizia intellettuale con un coetaneo, Antonio Ranieri, e un deludente rapporto amoroso con Fanny Targioni Tozzetti.

Il ciclo consiste in 5 componimenti poetici (\emph{Il pensiero dominante, Amore e Morte, Consalvo,} e \emph{A se stesso}), molto diversi dai \emph{Grandi Idilli} per una forma piú aspra; mancanza di quelle immagini ``vaghe ed indefinite''. Il linguaggio si fa aspro, eroico, combattivo, antimusicale, la sintassi complessa e spezzata.
\chapter{Testi}
\section{Lettere}
\subsection*{Lettera a Pietro Giordani 1819}
``Sono cosí stordito dal niente che mi circonda..."

É chiaro in questa citazione come la attivitá creativa di Leopardi sia stata frenata aspramente nell'anno 1819. In quell'anno, Leopardi aveva fallito un tentativo di fuga da Recanati, facendo risorgere quella perenne feritá aperta che era il sentimento di prigionia in un luogo sí isolato e arretrato quale era Recanati ("il niente che mi circonda"), privo di alcuna persona con cui discutere intellettualmente, neppure di alcune risorse per alimentare una mente moderna e romanticista. In aggiunta alla amara sconfitta, si aggiunge una malattia agli occhi, frenando quasi completamente la sua attivitá non solo di lettura, ma anche di produzione letteraria. Questi fatti rendono il 1819 un anno stagnante di inattivitá "stordente", poco diverso dalla morte; "non vedo piú il divario tra la morte e questa mia vita".
\subsection*{Lettera a Pietro Giordani 1820}
"Sto anch'io sospirando caldamente la bella primavera\dots"

Con il ritorno della primavera vi é in Leopardi un generale ritorno alle forze; lo invigoriscono i rumori e i colori della primavera, restaurando la sua possibilitá di immaginazione, e quindi un fiorire delle illusioni che rendono bella la vita. Gioisce estatico al ritorno della capacitá di illudersi, che lo aveva consolato negli anni precedenti. É in questa lettera che vediamo nascere la teoria del piacere che vedremo apparire nello Zibaldone a Luglio del medesimo anno; é qui, tra le lettere che scrive a Pietro Giordani, unica persona con cui Leopardi riesce a confidare, che germogliano le idee di Leopardi.

"e vedendo un cielo puro e un bel raggio di luna, [\dots] mi si svegliarono alcune immagini antiche\dots''

Come questa lettere anticipa la considerazione filosofica della teoria del piacere, nello stesso modo anticipa anche concetti come l'antico, l'idea di un passato remoto e offuscato dal passare del tempo, e reso piú bello; quanto anche l'immagine della Luna.

"Perché questa é la miserabile condizione dell'uomo, e il barbaro insegnamento della ragione, che i piaceri e i dolori umani essendo meri inganni, quel travaglio che deriva dalla certezza della nullitá delle cose, sia sempre e solamente giusto e vero."

In questo primo periodo, Leopardi é ossessionato con l'antagonismo tra ragione e natura.  Ha compreso ora, come spiegerá nello Zibaldone, che, essendo la felicitá a cui aspirano gli uomini una felicitá infinita e incolmabile, che ogni appagamento della felicitá porta ad una insoddifazione ed una ricerca piú assetata di essa, nessuno potrá mai essere felice. Quel "filosofastro" che cerca la felicitá nella veritá, non é altro che un prigioniero di quel circolo vizioso che Leopardi ha giá identificato.
\end{document}