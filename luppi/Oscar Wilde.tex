\documentclass{article}

\usepackage{tikz}
\usepackage{parskip}
\usepackage{amssymb}
\usepackage[utf8]{inputenc}
\usepackage{amsmath, empheq}
\usepackage[margin=3.75cm]{geometry}
\definecolor{mycolour}{RGB}{46,52,64}
\pagecolor{mycolour}
\color{white}

\title{\jobname}
\author{Eugenio Animali}

\begin{document}

\maketitle

\section*{Recap chapter 2}
Beautiful young aristocrat, Dorian gray. Artist Basil Hallward decides to paint him. Lord Henry is an art collector and wants to buy the painting, but basil says its not for sale, ad it is property of Dorian. Key moment when Dorian sees the painting: He loves it because he becomes finally aware of his beauty (it is the work of art and the artist who highlight beauty in the world - according to aestheticism and decadentism, the artist is the creator of beautiful things), but he is melancholy for he knows that he will not always be this beautiful (confirmation of the conversation about mortality they had recently before). Dorian makes a wish- if only it were the picture that grows old instead of me while i stay young and beautiful forever.
\section*{Chapter 20}
Dorian sits on a sofa in the countryside and considers what has happened. He had been a terrible person since when he had been able to escape punishment from his sins. He realizes that he did in fact enjoy his sins; this fears him for the future of his own control on himself. He had recently decided to pass on seducing a girl. He wishes that he had been punished at the moment for his actions, rather than be free to degrade himself continually. ```Smite us for our iniquities' should be the prayer of a man to a most just god."

``What was youth at best? A green, an unripe time, a time of shallow moods and and sickly thoughts. Why had he worn its livery? Youth had spoiled him.''

``Nor, indeed was it the death of Basil Hallward that weighted most upon his mind. It was the living death of his soul that troubled him.''

Slowly Dorian changes direction, removing any fault from himself: Basil is responsible for Dorian's troubles; everyone else died not by Dorian's hand. The murder had only been the madness of a moment.

``Confess? Did it mean that he was to confess? To give himself up, and be put to death? He laughed. He felt that the idea was monstrous. Besides, even even if he did confess, who would believe him? There was no trace of the murdered man anuwhere.'' but ``There was a God who called upon men to tell their sins to earth as well as to heaven. Nothing that he could do would cleanse him till he had told ihis own sin. His sin? He shrugged his shoulders.''

Dorian wants to get rid of the picture, which is the only evidence for his murder, so he stabs it with the same dagger that killed Hallward. By doing this he dies, and is found by the servants, and some men walking by alert some policemen. Unable to enter from the door, the servants enter from the roof to the balcony. They found a beautiful portrait of their master on the wall and an old man with a knife in his heart.
\section{Preface}
Preface dedicated to Walter Pater, university professor:
The artist is the creator of beautiful things.

To reveal art and conceal the artist is the point of art.

The critic is he who can translate into another manner or a new material his impression of beautiful things.

Those who find ugly meanings in beautiful things are corrupt without being carming. This is a fault. Those who findd beautiful meanings in beautiful things are cultivated. For these there is hope. They are the elect to whom beautiful things mean only Beauty.

There is no such thing as a moral or an immoral book. Books are well written or badly written. That is all.

The nineteenth-century dislike of realism is the rage of Caliban seeing his own face in a glass.

The nineteenth-century dislike of Romanticism is the ragge of Caliban not seeing his own face in a glass.

The moral life of man forms part of the subject-matter of the artist, but the morality of art consists in the perfect use of an imperfect medium.

No artist desires to prove anything. Even things that are true can be proved.

No artist has ethical sympathies. An ethical sympathy in an artist is an unpardonable mannerism of style.

No artist is ever morbid. The artist can express everything.

Thought and language are to the artist instruments of an art.

Vice and virtue are to the artist materials for an art.

From the point of view of form, the type of all the arts is the art of the musician.

From the point of view of feeling, the actor's craft is the type.

All art is at once surface and symbol.

Those who go beneath the surface do so at their peril.

Those who read the symbol do so at their peril.

It is the spectator, and not life, that art really mirrors.

Diversity of opinion about a work of art shows that the work is new, complex, and vital.

When critics disagree the artist is in accord with himself.

We can forgive a man for making a useful thing as long as he does not admire it.

The only excuse for making a useless thing is that one admires it intensely.

All art is quite useless.

OSCAR WILDE.
\end{document}