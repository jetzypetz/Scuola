\documentclass{article}

\usepackage{tikz}
\usepackage{parskip}
\usepackage{amssymb}
\usepackage[utf8]{inputenc}
\usepackage{amsmath, empheq}
\usepackage[margin=3.75cm]{geometry}
\definecolor{mycolour}{RGB}{46,52,64}
\pagecolor{mycolour}
\color{white}

\title{\jobname}
\author{Eugenio Animali}

\begin{document}
\maketitle

\section{Introduzione}
\subsection{La societá é in crisi}
\begin{enumerate}
    \item Caduta Impero Napoleonico
    \item 1815 $\to$ Congresso di Vienna
    \item Periodo di Restaurazione; Diffusione idee nazionaliste
    \item Ritorno al passato
\end{enumerate}
\subsection*{Caratteristiche pittore Romanticista}
\begin{itemize}
    \item Nasce figura artista povero- artista dipinge per esprimersi, non per vendere; artista non concorda con filosofia moderna.
    \item Rifiuto del presente- per ripulsione da esso (se mai rappresentato, é in modo dispregiativo).
    \item Stile Pittoresco- Dipinti dell'infanzia, bellezza dei ricordi
    \item Affronto all'illuminismo- Utilizzo della ragione non ci ha portati ad un mondo migliore. Studio dell'irrazionate
    \item L'opera d'arte educa- Parallelismo al Neoclassicismo(In quadri storici). Ma in modo emotivo.
    \item Consapevolezza dell'inesistenza della perfezione- rappresentazione del brutto- riscuote le coscienze
    \item Natura che suscita emozioni.
\end{itemize}
\subsection*{Protagonisti}
\begin{itemize}
\item Emozioni e Sentimenti - maggiore coinvolgimento dello spettatore, entriamo nell'opera stessa
\end{itemize}
\section{Théodore Géricault}
\subsection{Vita}
\begin{itemize}
    \item Nasce a Ruen, da famiglia agiata
    \item Parigi: studi indipendenti, conosce Delacroix (pittore autonomo)
    \item Roma: viaggio studio di un anno (non vince premio di Roma)
    \item Muore 33 anni - non ha la possibilitá di esprimere al meglio il romanticismo storico
\end{itemize}
\subsection{stile}
\begin{itemize}
\item Scelta di temi classici o contemporanei
\item Amore per i Cavalli
\item Studio Corpo umano
\item Bravo a disegnare
\end{itemize}
\subsection{Esposizione al Salon 1814}
\subsection{Zattera della Medusa}
\subsubsection{Romantico}
\begin{itemize}
\item pieno dell'azione
\item brutto della realtá
\end{itemize}
\subsubsection{Neoclassico}
\begin{itemize}
\item anatomia perfetta
\item stesura del colore
\item Piramide
\end{itemize}
\subsection{Alienati}
\begin{itemize}
    \item Studio psicologico di persone alienate.
\end{itemize}
\section{Eugéne Delacroix}
\subsection{Vita}
Nasce a Charenton - Saint Maurice in Francia settentrionale

artista indipendente - non formato nella accademia delle belle arti. Dipinge e impara da solo, andando a studiare le opere del Louvre. Da Michelangelo prende lo studio del corpo; da Tiziano prende lo studio del colore (studiava prima il colore e poi il disegno; tirava fuori il disegno dal colore- niente linea di contorno. Verso la morte era cieco e stendeva il colore con le mani. Colorista) anche Delacroix non avrá contorni;
\subsection{stile}
Romantico Storico

Passato di interesse é medievale
\section{Opere}
\subsection{La barca di Dante}
Ottavo canto dell'inferno.

Disegni un po classici, riprendono i nudi di Michelangelo. Struttura piramidale senza equilibrio perché in basso, tutto é movimento e ondulato.

Colori:
\begin{itemize}
\item chiaroscuro
\item colori accesi
\item mette direttamente i colori sulla tela, senza mischiarli prima sulla tavolotta.
\item non ci sono contorni, e i corpi si confondono con il mare.
\end{itemize}
\subsection{Il massacro di Scio}
prima esposizione succede al salon. Il salon nasce nel fine 1600 per artisti nuovi, per esposizioni temporanee. Poi diventa publica verso fine 1700, ma vengono rifiutate sempre piú opere. Napoleone 3 crea il salon des refusees.
questa opera era una denuncia di un massacro da parte dei turchi di 20,000 greci dopo delle ribellioni per indipendenza.

Non ci sono simmetrie, o perfezioni, ma l'occhio passa da una parte all'altra. Lo spettatore é quindi il protagonista, che puó scegliere cosa guradare, senza essere indirizzato dall'artista verso punti di fuga. Quindi fu criticato molto aspramente.
\subsection{La libertá guida il popolo}
Opera storica che é diventata il simbolo della rivoluzione, ma non parla della rivoluzione Napoleonica. Parla di un insorgimento contro re Carlo X di borbone.

riferimento a Gericault la zattera di Medusa: uomo seminudo con calzino in fondo.

Folla di persone avanza verso lo spettatore, che indica un coinvolgimento dello stesso. Sono persone di diversi ceti sociali: Donna popolana, un borghese, un bambino povero\dots Donna al centro é il primo nudo contemporaneo! Sappiamo di stare a parigi per notre dame in sfondo.

Di Classico: Forma piramidale. Donna al centro con seno sodo.

Non Neoclassico: Nel pieno dell'azione. Spettatore ne fa parte.
\subsection{Il rapimento di Rebecca}
Ripreso dal Medioevo, da Ivanhoe. Rapimento di Rebecca da parte dei Saraceni.

Di Romantico: Opera in pieno movimento. Linee ondulate. In sfondo un furor di persone che scappano da un castello in fiamme che cade a pezzi- studio del fumo. Cavallo fatto a macchie. 
\end{document}