\documentclass{article}

\usepackage{tikz}
\usepackage{parskip}
\usepackage{amssymb}
\usepackage[utf8]{inputenc}
\usepackage{amsmath, empheq}
\usepackage[margin=3.75cm]{geometry}
\definecolor{mycolour}{RGB}{46,52,64}
\pagecolor{mycolour}
\color{white}

\title{\jobname}
\author{Eugenio Animali}

\begin{document}
\maketitle
\section{Vita}
Saragozza (forse esiliato, forse per fare grand tour) $\to$ Roma $\to$ Bordeaux

29 anni $\to$ vicedirettore di pittura all'Accademia di S. Ferdinando

53 anni $\to$ pittore ufficiale di corte

\section{Aspetti Neoclassici}
Porta avanti idee dell'illuminismo. Non rappresentando la perfezione ma la bestialitá umana.

\section{Opere}

\subsection{Capricci}
Serie di 80 opere grafiche.
scopo educativo- espongono i vizi, le superstizioni, le bassezze umane

\subsubsection{Que viene el coco}
superstizione

\subsubsection{Que se llevaron}
Rapimento

\subsubsection{A caza de dientes}
rubare i denti d'oro agli impiccati

\subsubsection{Il sonno della ragione genera mostri}
diverse ipotesi:
\begin{enumerate}
    \item L'allontanamento dalla ragione
    \item Il sogno della ragione della rivoluzione francese ha dato vita a dei mostri
    \item L'artista che sogna
\end{enumerate}
\subsection{Maja vestita e maja desnuda}
Donna non divina nuda. Consapevole della sua bellezza.

\subsection{Le fucilazioni del 3 maggio 1808}
Soldati francesi visti di spalle - niente volto e al buio (simbolo della razionalitá), disumanizzati. Al centro vi é una lanterna che illumina parte sinistra del dipinto. Personaggio in ginocchio che chiede di essere salvato. Corpi morti in basso. Volti terrorizzati a sinistra. Sfondo di Madrid.

\subsection{Omino che cade sulle scale}
Comico, tragico. Fa vedere il tragico.

\subsection{Gli sterratori}
Scavano la fossa per un morto. Episodio brutto, tragico.
\end{document}