\documentclass{article}

\usepackage{tikz}
\usepackage{parskip}
\usepackage[utf8]{inputenc}
\usepackage{amsmath, empheq}
\usepackage[margin=3.75cm]{geometry}
\definecolor{mycolour}{RGB}{46,52,64}
\pagecolor{mycolour}
\color{white}

\title{Napoleone}
\author{Eugenio Animali}
\date{22 09 2022}

\begin{document}

\maketitle

\section{Introduzione}

Il Papa é contro la rivoluzione perché é ancora legato a concetti medievali di leggitimare gli Imperatori con il processo divino

\section*{Come leggitimare Potere}

\subsection*{Problemi}

Giuseppina é troppo grande per essere erede.

Nessun legame con alcuna casa nobile.

\subsection*{Soluzioni}

Si lega alla casa Asburgo sposandosi con Maria Luisa d'Asburgo (nipote di Maria Antonietta, Figlia di Francesco II), con grande rammmarico per la lasciata Giuseppina. - Problemi per la uccisione di Maria Antonietta Asburgo... \color{red}figlio della rivouzione + imparentato con la piú importante famiglia nobile\color{white}

Anche se sciolto il sacro romano impero germanico, gli asburgo avevano comunque tutto il potere.

Dopo un anno nasce un erede maschio, che viene incoronato Re di \underline{Roma} - Tutto il potere imperiale si ricollega sempre a Roma.

1810-12 Periodo di pace. Ci sono diversi gruppi di persone non contenti. Giacobini per ritorno all'impero; Nostalgici Borbonici per allontanamento da Borboni; Cattolici per incarceramento Papa; - Scoppio con Fichte, filosofo che esalta popolo tedesco perche lingua tedesca non é stata corrotta dalla lingua latina. - mantiene la sua purezza culturale.

Gigantesca armata verso la russia. 700,000 uomini transnazionali, tutti popoli sottomessi da Napoleone.Grande Armée

Tattica russa della terra bruciata, non rimane niente agli europei, proteggono solo Mosca, vuota, mentre lo Tzar scappa in Siberia. Arrivati a Mosca, si svegliano in una Mosca in fiamme, perché i russi hanno nascosto barili di polvere da sparo in case sparse per la cittá.

A ottobre Napoleone decide di ritirare, marciando nella steppa russa innevata, con i Kosacchi che attaccavano e scappavano in continuazione. Tornano 100,000 uomini.

Colpo di grazia ad Ipsia, Germania contro VI coalizione (indeboliti) - Ottobre 1913. Deve abdicare al trono.

\section{Restaurazione}
\subsection{Prima}
Nazione $\to$ Sovranitá $\to$ costituzione

Ugualianza

libertá

impero
\subsection{Ora}
monarchia

Assolutismo

Tornano i Borbone con Luigi XVIII. Non vogliono umiliare Napoleone perché mantiene tanto consenso (perso solo due volte: L'Ipsia e Russia) - donata casa all'elba

Ad una festa di carnevale 1815, Napoleone scappa da elba, passando le navi inglesi. Sbarca in Francia. L18 manda un esercito. Napoleone apre le braccia "Eccomi, Sono il vostro Imperatore, sparatemi se volete". Esercito si schiera con lui e marciano a Parigi. Scappa L18.

VII coalizione (Inglesi e prussiani) a Waterloo. Napoleone stava per vincere. Manda parte dell'esercito dai Prussiani, per non farli unire, mentre parte maggiore attacca Inglesi. Stanno per prevalere sugli Inglesi di Wellington. "Datemi Blucher o la notte" 18 Giugno. Blucher riesce a sganciarsi e arriva sul campo di Waterloo, schiacciando Napoleone. Napoleone chiama anche la vecchia guardia ma si devono arrendere. Napoleone chiede di passare la sua prigionia in territorio Inglese. Mandato a Sant'Elena. Ei Fu.
\section{Congresso di Vienna}
Emfasi su andare contro idea dell'impero.

Idea di Concertazione- Mantere regini assoluti ovunque per non far cadere in rivoluzione nessun paese.

Congresso di vienna per restaurare poteri in europa, restaurare anche i principi precedente. Cancelliere austriaco Metternich cerca di realizzare un equilibrio tra le potenze; un ``concerto tra le nazioni''. Gran bretagna, austria, russia e prussia sono le 4 nazioni che vinsero Napoleone. Francia non subisce forti decurtazioni territoriali per due motivi:

\begin{enumerate}
    \item Talleyrand fa ragionamento- la colpa non é della francia- francia é una vittima dei movimenti di Napoleone.
    \item Rappresentanti degli altri stati europei si rendono conto che non possono umiliare la francia perche Napoleone ancora riscuote grande successo- paura di farli arrabbiare
    \item L18 Concede una Charte octroyée - carta di concedimenti nei confronti dei bonapartisti.
\end{enumerate}

Principi: equilibrio e legittimitá
\subsection{Cosa succede}
Spagna e Sardegna diventano stati cuscinetto.

Nasce confederazione germanica: Include Boemia, Austria, e Trentino Alto Adige.

Altri stati del nord sono imparentati con germani, quindi indirettamente influenzati

Nasce la Santa Alleanza:La russia, la prussia e l'austria. Obbiettivo di garantire equilibrio europeo e intervenire in due casi fondamentale:
\begin{enumerate}
    \item se ci sono rivoluzioni che vogliono cambiare lordine costituito
    \item intervenire se ci sono moti contro l'ancienne regime
\end{enumerate}
Gran bretagna non aderisce alla sant'Alleanza perche non piacciono toni mistici -> Quadruplice alleanza: santa alleanza e inghilterra- bisogna impedire che la francia ritorni una grande potenza.

\subsection{Costituzione di Cadice}
durante l'assedio Napoleone 1812 spagnoli scrivono una costituzionale dovi si vuole affermare una monarchia costituzionale liberale (diritti civili).
\section{Societá Segrete}
Ribellano alla restaurazione. Gerarchiche. Solo i maestri sanno i fini ultimi della Societá Segreta. Ne fanno parte membri della medio alta borghesia o nobili di mentalitá illuminista o militari della rivoluzione. Lavorano in modo clandestino, e hanno in genere idee liberali (suffragio censitario). Carboneria era peró piú democratica(a differenza della massoneria). Affiliati sono Media alta borghesia perché devono pagare l'entrata, studenti universitari e ufficiali dell'esercito.

Nasce la massoneria in inghilterra nel 1700, ma durante la restaurazione diventa una associazione politica che difende i diritti civili.

\subsection{Moti del 20-21}
Costituzione di riferimento é costituzione di Cadice. Prima scintilla scoppia a Cadice nel 1 gennaio 20. Nel porto di Cadice sono raccolti molti soldati per andare in america latina (che si stanno rendendo indipendenti dal dominio spagnolo- Garibaldi andrá a combattere). Sollevazione delle truppe spagnole, spinte da societá segrete, che richiedono che venga ridestinata la Costituzione di Cadice. Re é costretto a seguire gli ordini e Spagna diventa di stampo liberale.

Sud Italia 1 Luglio 1820 a Nola: ammutinamento di una guarnigione che aveva combattuto con Napoleone. Costituzione di Cadice affermata a Nola.

15 Luglio 20 in Sicilia: insorti chiedono costituzione come a Cadice, ma con indipendenza da Napoli. Siccome moto siciliano richiede forte autonomia della sicilia, il governo Napoletano degli insorti manda un esercito ad attaccare insorti di Sicilia, perche non sono daccordo con autonomia. Motivo di fallimento delle insurrezioni- battaglie tra insorti. Altro motivo: battaglie tra insorti democratici e liberali. Ferdinando chiede intervento della santa alleanza e quella interviene anche in sud italia, soffocando i moti del 20 21.

Marzo 21 Ferdinando poté ritornare sul suo trono.

In piemonte scoppia un moto in ritardo marzo 21. Re vittorio Emanuele I, pur di non concedere la costituzione di Cadice, abdica a favore del fratello Carlo Felice, che non é a Torino, e la reggenza passa al nipote Carlo Alberto, di tendenze liberali, e concede uno statuto dove vengono affermati moderati diritti civili. Quando Carlo Felice torna, sconfessa e ritira lo statuto, mettendo in atto una forte repressione.

Moto `decabrista' (dicembre) anche in Russia in occasione di giuramento del nuovo tzar Nicola I. Represso rapidamente tipico di autocrazia. Dicembre 25

Manca un coordinamento nazionale e internazionale

Grande massa del popolo rimane estranea a questi moti - erano solo gli ufficiali, ed elite intellettuali isolati, perché serve una base di educazione per capire diritti sociali.

L'unico moto che ha successo é in Grecia perché una parte importante della popolazione ne partecipa. Turchi ottomani (il grande malato d'europa) rispondono violentemente - stragi di villagi. Interviene la russia con la scusa di proteggere greci ortodossi come loro. In realtá interviene per mettere un piede nel mediterraneo. Per non lasciare il campo libero alla russia, Francia e Inghilterra si alleano alla Grecia. Battaglia sotto acque di Navarino 20 ottobre 1827. Ottomani vengono distrutti. Prima crepa nella Santa Alleanza perche russi vanno da soli in guerra.
\subsection{Moti del 30-31}
Organizzati di nuovo da Societá Segrete. Scoppiano in francia nel 1830 a causa del comportamento del re Carlo X conservativo (1824). Aveva riportato le nobiltá al loro potere vecchio. tutto parte dalle 4 ordinanze: abolisce Charte octroyée, restringe libertá di stampa, diritto di voto, e scioglie parlamento per fare altre elezioni.

Carlo X fugge, francia diventa monarchia costituzionale, Luigi Filippo d'Orléans concede una costituzione accenttando di regnare non piú per grazia di Dio, ma per volontá della nazione.
\subsection{Indipendenza Belgio}
Legati ad olanda per monarchia, ma molto diversi tra loro. Olandesi peró erano una classe di potere che opprimeva i Belgi
\begin{enumerate}
\item Religione: Olandesi Luterani, Belgi Cattolici
\item Lingue: Olandesi parlano Fiammingo, Belgi Francese
\item Economia: Olandesi Commercianti (libero scambio), Belgi Agricoli (protezionismo)
\item Politica: Olandesi hanno tutto il potere
\end{enumerate}
Francia (LF d'O) interviene con il Belgio per allargare la propria influenza. gennaio 1831 diventa indipendenti.
\subsection{Polonia}
Polonia era stata spartita tra russia prussia e austria. Invidiosi di Belgio aiutato da Francia. Russia risponde con spietata repressione (russificazione). Lingua Russa imposta
\subsection{}
La germania supera l'inghilterra verso gli anni 1860 a livello economico, perche pur venendo dopo acquistano i machinari piú moderni, mentre gli inglesi stanno ancora indietro, per che i macchinari sono investimenti a lungo termine.
\pagebreak

la7 mercoledi una giornata particolare
\end{document}