\documentclass{article}

\usepackage{tikz}
\usepackage{parskip}
\usepackage{amssymb}
\usepackage[utf8]{inputenc}
\usepackage{amsmath, empheq}
\usepackage[margin=3.75cm]{geometry}
\definecolor{mycolour}{RGB}{46,52,64}
\pagecolor{mycolour}
\color{white}

\title{\jobname}
\author{Eugenio Animali}

\begin{document}
\maketitle
\section{Prima Rivoluzione}
\begin{itemize}
\item Carbone
\item Capitalismo
\item Tessile
\item Macchina a Vapore
\item Treno
\end{itemize}
\section{Secondi Arrivati}
\begin{itemize}
\item Francia
\item Usa
\item Belgio
\item Territori Germani
\end{itemize}
\section{Seconda Rivoluzione(entrano in gioco i Late Comers)}
\begin{itemize}
\item Italia
\item Spagna
\end{itemize}
\subsection{Caratteristiche}
\begin{itemize}
\item Acciaio - Esercito
\item Motore a Scoppio - Macchina
\item Elettricitá - Luce
\item Farmacia
\item Comunicazioni
\end{itemize}
Impatto reale sulla vita delle persone:
\begin{itemize}
\item ascensore
\item tram
\item bicicletta
\item telegrafo
\item Great Exhibition
\item Olimpiadi
\item Tour Eiffel
\end{itemize}
nascono 3 nuovi filoni politici:
\subsection{socialista}

concetto chiave é giustizia sociale. riorganizzare la proprietá privata a favore delle classi piú povere. - Ridurre differenze.

Va superato il modello capitalista che é causa delle differenze sociali grazie al suo meccanismo di competizione, anche se ci sono discussioni su com farlo.

Approccio internazionalista. Problemi non sono relativi alla nazioni ma tutta la classe operaia.

Nasce nel 1864 la prima internazionale socialista, dove ci sono 2 filosofie preponderanti: comunismo di marx e anarchismo di bakunin.
\begin{itemize}
\item per bakunin, il tema principale é abolizione della concezione di stato. Ha piú risonanza subito, poi piú cresce l'industria, e quindi la classe operaia, piú cresce la concezione comunista.

\item per Marx bisogna cambiare il sistema economico, che é la base dei problemi.
\end{itemize}
Nel 1866 finisce la prima internazionale.

1889 Nasce seconda internazionale, dopo morte di Marx, che dichiara ispirazione Marxista, e quasi tutti i presenti sono comunisti. Interpretazioni diverse di filosofia marxista.
\begin{itemize}
\item Marxismo Ortodosso
Marxismo ortodosso di Karl Kautsky, Leader dell'SPD. Idea di base é che i movimenti comunisti non sono ancora pronti per la rivoluzione, e bisogna attendere, per far crescere forze comuniste creando partiti politici, facendo propaganda, e guadagnando voti.

\item Bolsceviti: Lenin dice che bisogna prima prendere il potere, poi ci si occupa di accrescere il consenso.
\end{itemize}
\subsection{Revisionismo di Bernstein}
Bernstein rivede le teorie marxiste. Sostiene che marx ha sbagliato su un aspetto: che il capitalismo é destinato inevitabilmente alla crisi irreversibile. Secondo Bernstein, il capitalismo é destinato a crescere e quindi non bisogna riporsi il tema rivoluzionario; bisogna porsi in modo da collaborare con il capitalismo e non ci sono le condizioni per una rivoluzione o una presa del potere, ma bisogna lentamente collaborare con politici per dare piú diritti delle classi operaie.

\end{document}