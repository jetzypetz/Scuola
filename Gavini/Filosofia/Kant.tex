\documentclass{article}

\usepackage{tikz}
\usepackage{parskip}
\usepackage{amssymb}
\usepackage[utf8]{inputenc}
\usepackage{amsmath}
\usepackage[margin=3.75cm]{geometry}
\definecolor{mycolour}{RGB}{46,52,64}
\pagecolor{mycolour}
\color{white}

\title{\jobname}
\author{Eugenio Animali}

\begin{document}
\maketitle

\section{Siamo d'accordo con Hume che la metafisica non é una conoscenza valida?}
Va piú in profonditá, per capire perché, e alla fine - devo studiare

\section{Esaminiamo la Scienza}

\subsection{Su cosa si basa la conoscenza?}
Prima di Kant vi erano state due soluzioni fondamentali:
\begin{enumerate}
    \item Giudizi ANALITICI A PRIORI
    \begin{itemize}
        \item eg. I corpi sono estesi
        \item razionalisti
        \item Non ampliano la conoscenza - pretendono di parlare di ció che é necessario; informazione giá interna al concetto di corpo (analuo il concetto)
        \item Deduttivi
        \item Fondamento: Principio di non contradizione
    \end{itemize}
    \item Giudizi SINTETICI A POSTERIORI 
    \begin{itemize}
        \item eg. I corpi sono pesanti
        \item Empiristi
        \item Sono FECONDI - il predicato aggiunge qualcosa al soggetto perche il corpo non include il concetto di pesantezza.
        \item A POSTERIORI perché si devono provare; e dopo si puó affermare.
        \item Induttivi
        \item Fondamento: Esperienza
    \end{itemize}
\end{enumerate}
Kant prende gli aspetti positivi di entrambi:
\subsection{Giudizi SINTETICI A PRIORI}

"Benché ogni nostra conoscenza cominci dall'esperienza empirica, da ció non segue che essa derivi interamente dall'esperienza."
\begin{itemize}
    \item eg. Tutto ció che accade ha una causa
    \item eg. $5+7=12$
    \item eg. "In tutti i cambiamenti del mondo corporeo, la quantitá di materia non cambia. nel concetto di materia, io non penso permanenza, ma solo la sua estenzione nello spazio. Percio io oltrepasso realmente il concetto di materia  per aggiungervi a priori qualche cosa che in quel concetto non pensavo."
    \item sono FECONDI perche concetto finale non é compreso nel concetto iniziale
    \item ma FONDAMENTALI E UNIVERSALI
    \item Vuole giustificare Scienza di Newton.
    \item Dai razionalisti prende il valore universale delle leggi- Scienza deriva da strumenti conoscitivi innati
    \item Dagli Empiristi prende la necessaria derivazione della scienza dalle dimostrazioni empiriche
    \item Fondamento: ???
\end{itemize}
    
Che cosa é quel concetto di incognita ``x"?...

\section{Rivoluzione Copernicana}
Come coperinico mise il sole al posto della terra, io metto l'oggetto al posto del soggetto:

`La veritá a Priori sta nel soggetto e non l'oggetto'

Prima: soggetto deve studiare la realtá - problema con razionalisti e Empiristi perché PANTA REI

Ora: Non é la mente che si deve modellare attorno a queste conoscenze. É la realtá che si deve modellare con le nostre forme a priori della nostra mente.

Non é che la mente degli uomini scopra le leggi nascoste della natura, ma la mente degli uomini impone alla realtá stessa queste leggi della scienza prodotte dalle forme a priori della mente.

Popper "dobbiamo abbandonare l'opinione secondo cui siamo dei soggetti passivi sui quali la natura imprime la propria regolaritá."

Galileo: ``il grande libro della natura é scritto in caratteri matematici e triangoli e cerchi'' Kant risponde che non possiamo sapere com'é fatta la realtá, perché il Noumeno é nascosto a noi.
\subsection{Forme della mente A PRIORI}
\subsubsection{Intuizioni pure/ forme della semsibilitá}
\begin{enumerate}
    \item tempo
    É piú importante dello spazio perché tutti i nostri sentimenti hanno una collocazione nel tempo ma non tutte nello spazio- i sentimenti, le intuizioni non hanno spazio ma hanno sempre tempo.
    \item spazio
    ``LO SPAZIO NON É UN CONCETTO EMPIRICO RICAVATO DA ESPERIENZE ESTERNE. Infatti, affinché certe sensazioni vengano riferite a qualcosa fuori di me, deve esserci giá a fondamento la rappresentazione dello spazio.''
\end{enumerate}
Senza di esse non potremmo mettere gli eventi in relazione per tempo e spazio. Diverso da Locke, che vedeva l'idea tempo e spazio che vengono dopo l'esperienza, per ordinarli. Diverso da Newton, che vedeva l'idea di tempo e spazio come entitá ontologiche assolute, che ci sarebbero anche se non ci fosse il mondo. Kant risponde a Newton ``come fai a concepire qualcosa che senza un oggetto reale, sarebbe comunque reale'' Per Kant sono forme a priori del soggetto quindi soggettivo per tutta l'umanitá.
\subsubsection{Dodici categorie dell'intelletto}
In aristotele le categorie avevano un valore sia ontologico che gnoseologico perché per aristotele vi era una corrispondenza tra logica ed essere. Per Kant hanno solo valore logico. Valgono solo per il fenomeno; non per il Noumeno.

Come ci arriva? Pensare significa giudicare; giudicare significa attribuire un predicato ad un soggetto. Esistono tante categorie quante le possibilitá di attribuire un predicato ad un soggetto.

\begin{enumerate}
    \item quantitá
    \begin{enumerate}
    \item unitá
    \item pluralitá
    \item totalitá
    \end{enumerate}
    \item relazione
    \begin{enumerate}
    \item sostanza e accidente
    \item causa ed effetto (il sole scalda la pietra) Noi imponiamo le nostre regole alla natura (cosí é daccordo in parte con Hume. Contro hume perché entro dei limiti la scienza di newton funziona.)
    \item azione reciproca
\end{enumerate}
    \item \dots
\end{enumerate}

\begin{enumerate}
    \item \dots
    \item Geometria
    Geometria é legittimata perché é legata alla idea assoluta di spazio.
    \item Aritmetica
    Poggia sul tempo - successione di numeri poggia su successione di istanti.
\end{enumerate}
\subsubsection{qualitá}
\begin{itemize}
\item Innate
\item Uguali in tutti gli uomini
\item $\to$ usate nello stesso modo tra gli uomini- creano una scienza universale e necessaria e valida.$\to$ Massimo esponente dell'illuminismo- egalité nell'aspetto gnoseologico.
\end{itemize}

\section{Innatismo Formale}
Platone- reminescenza - sono Innati i contenuti della conoscenza

Kant- sono Innate le \underline{forme} della mente (ció attraverso cui conosciamo)
\section{Critica alla ragion Pura}
Nei limiti in cui la ragione puó agire, la ragione é la autoritá massima.


Serve una conoscenza che sta dentro alluomo, perché deve essere a priori. Quindi deve essere valida per tutti gli uomini. 
\subsection*{Le facoltá della conoscenza (partizione della Pura)}
\begin{enumerate}
    \item sensazione - é la facoltá con cui gli oggetti ci sono dati intuitivamente attraverso i sensi, ordinati tramite le forme a priori dello spazio e del tempo
    \item intelletto- é la facoltá con cui pensiamo i dati sensibili tramite le 12 categorie dell'io penso. Conseguenza é l'io Legislatore, che puó produrre giudizi sintetici a priori sulla natura (fisica). Ecco Rivoluzione Copernicana, che separa Fenomeno e Noumeno.
    \item ragione - Ragionare su una causa al di lá dell'esperienza, che non vediamo. É la facolta attraverso cui, procedendo oltre l'esperienza, cerchiamo di spiegare globalmente la realtá mediante le idee di ``anima'', ``mondo'' e ``Dio'', che apparterrebbero al noumeno. Idea di mondo nasce dal tentativo prodotto dal desiderio umano di descrivere il mondo con una legge generale ed unica. Idea di Mondo ha creato 4 antinomie - fare domande tipo l'uomo é libero? non avrá mai risposte, perché provengono dalla sfera Noumenica. Idea di Anima é necessaria da ragion pratica. Se non veniamo premiati per agire moralmente, deve esistere dopo la vita. E deve esistere una essenza che ci dia questa ricompensa.
\end{enumerate}
\subsection{L'estetica trascendentale}
Studia la sensibilitá e le forme a priori di tempo e spazio.
\subsection{L'analitica trascendentale}
Kant vuole comporre l'intelletto e analizza le forme a priori dell'intelletto. Senza di esse non avremo la facoltá di pensare l'oggetto. L'intelletto e la sensibilitá sono entrambi indispensabili alla conoscenza.

``Senza sensibilitá, nessun oggetto ci verrebbe dato e senza intelletto nessun oggetto verrebbe pensato. I pensieri senza contenuto [senza intuizione] sono vuoti, le intuizioni senza concetto sono cieche.''

Vedi le dodici categorie dell'intelletto.
\subsubsection{Concetti empirici e concetti puri}
Il concetto empirico é un concetto costruito attraverso il materiale empirico (la bottiglia [sia quando che quando non la sto sperimentando]; non sono sufficienti le Intuizioni pure di tempo e spazio).

``Al concetto puro non si mescola alcuna sensazione; ha origine solo nell'intelletto''
\subsubsection{La deduzione trascendentale}
Considerazione viene cambiata molto per la seconda edizione.

Deduzione in senso giuridico- Il fatto che io sia in possesso di una certa cosa non vuoldire che sono autorizzato, leggittimato a utilizzarla sulla realtá empirica. Cosa ci garantisce che la natura obbedirá alle nostre categorie? Questo problema non si poneva sulle forme di spazio e tempo perché senza di esse non ci arriverebbe niente.

``le condizioni soggettive del pensiero devono avere una validitá oggettiva. Qui emerge dunque una difficoltá che non abbiamo incontrato nel campo della sensibilitá: in qual modo, cioé le condizioni soggettive del pensiero debbano avere una validitá oggettiva, ossia ci diano le condizioni della possibilitá di ogni conoscenza degli oggetto''
\subsubsection{Soluzione Io Penso}
É il centro mentale unificatore di tutte le categorie. Le categorie sono le funzioni dell'io penso. L'io penso é la sintesi di tutte le categorie.

``L'io penso deve poter accompagnare tutte le mie rappresentazioni''

serve l'io penso per sapere che le rappresentazioni che ho fatto sono mie.
\section*{Definizioni}
Giudizi - Connessione tra un oggetto e un predicato

Fenomeno- TO PHAENOMENON - ció che appare; la realtá quale ci appare attraverso le forme della mente. La realtá azzurra

Noumeno - NOUS - la reatá in se. La reatá che si puó pensare, ma non sperimentare.

\section{Ragion Pratica}
Opera etica che riguarda il problema della morale universale. Come si delinea una morale universale? O é valida sempre o non é attendibile.

Non possiamo prendere la morale da Dio perché se non si puó ragionare su Dio, non puó derivarci un ragionamento. Si usa il ragionamento umano, che é universale.

\subsection{Le qualitá}
\begin{itemize}
\item Autonoma - la ragione risponde solo a se stessa
\item Formale - ci indica la regola generica, che poi va applicata concretamente da noi
\item Rigorosa - va sempre applicata nello stesso modo.
\item Intenzione - non importa l'effetto dell'azione ma solo l'intenzione
\end{itemize}
\subsection{Forme per affermare la morale}
\subsubsection{Massime di Vita}
"Io credo che..."

Sono soggettivi e quindi non possono produrre legge universale.
\subsubsection{L'imperativo ipotetico}
"Se vuoi, allora devi"

Imperativo é un comando che la ragione ci dá nel caso volessimo raggiungere un certo obbiettivo.

Questi sono applicabili universalmente e utili ma sono legati ad una condizione, e quindi guidano solo una parte delle nostre azioni. Serve una legge invece che possa risolvere tutte le nostre azioni.
\subsubsection{L'imperativo Categorico}
"Tu devi"

É categorico- comando a cui non sono permesse condizioni. É un comando che la ragione applica sempre a tutti. La morale Kantiana é fondata sul dovere.
\subsection{Il Dovere}
L'uomo morale si fa sempre guidare dal dovere, perché é l'unica via basata sulla ragione. Facoltá che esiste solo negli uomini. La tua azione per essere morale deve essere universalmente replicabile. ``Se tutti facessero questa cosa, il mondo sarebbe migliore o peggiore?''

``Agisci in modo da considerare l'umanitá, sia nella tua persona, sia nella persona di ogni altro, sempre anche al tempo stesso come scopo, e mai come semplice mezzo''

\subsection{I Postulati della Morale}
Tre veritá non dimostrabili ma necessarie per proseguire.
\begin{enumerate}
    \item Libertá
    \item Anima
    \item Dio
\end{enumerate}
se il dovere é definito da ció che é meglio per tutti, é necessariamente condizionato da un fine.
\section{Critica del Giudizio}
La realtá ha un finalitá dell'esistenza? L'uomo non arriverá mai dal punto di vista conoscitivo, ma vi é una facoltá di indagine, una esigenza naturale che non puó essere appagata.
\subsection{Giudizi Riflettenti}
Nascono dalla riflessione che l'uomo fa sui fini della natura. Sono lo specchio di un sentimento interiore. Non vanno a cercare una razionalitá, ma vanno comunque indagati perché ci fanno sentire delle sensazioni.
    \begin{enumerate}
        \item Teleologico - V'é un ragionamento scientifico, una concettualizazione. Natura oggettiva. ``Questo scheletro é fatto per reggere''
\item Estetico - Mancanza di ragionamento. Natura soggettiva. ``Questo tramonto sembra fatto per me''
\begin{enumerate}
    \item Empirici - rimandano ad una specificitá. Non é bello ció che é bello, é bello ció che piace. É Piacevole
\item Puri - che rimandano ad una universalitá. Secondo Kant esistono dei giudizi universali sulla estetica. 4 charatteristiche della bellezza universale. Un'altra rivoluzione copernicana: la bellezza la vediamo noi, non esiste in natura. Tutto ció che ha tutte queste qualitá é bello, e dovrebbe piacere a tutti:
\begin{enumerate}
\item Oggetto di un piacere senza interesse.
\item Oggetto di un piacere senza ragionamento.
\item Oggetto di un piacere necessario. Viene una volontá di affermare a tutti che é bello.
\item Il bello é la forma della finalitá di un oggetto senza la rappresentazione di uno scopo. Il bello non nasce da un certo canone estetistico. Una musica che ha lo scopo di adeguarsi ad un canone stilistico non é bello.
\end{enumerate}
\end{enumerate}
\end{enumerate}
\subsection{Il Sublime}
La bellezza rimanda ad una armonia; Il sublime rimanda ad un eccesso, o un caos. Esistono due forme:
\begin{enumerate}
\item Matematico - Nasce la dove siamo a confronto con delle grandezze sproporzionate. Quando guardi le stelle. Inadeguatezza $\to$ inizialmente negativo, poi realizziamo che ha un numero, la grandezza della natura é finita, a qual punto ci sentiamo meglio perché noi sappiamo pure cogliere l'infinito (vinciamo noi). All'inizio OPPRESSIONE schiaccia l'uomo, poi si ribalta in piacere LIBERTÁ quando l'uomo comprende che grazie alla sua immaginazione é superiore alla natura che ha dei limiti materiali. Oppressione e libertá sono solo possibili con la morale.
\item Dinamico - Nasce con lo scatenarsi della natura. Natura in tempesta. Impotenza
\end{enumerate}
\subsection{Il Genio - bellezza nell'arte (diretta da volontá umana)}
Il genio crea canoni nuovi e originali rompendogli schemi. Sono questi che gli altri seguono ma non riescono a riprodurre perfettamente perché il genio é colui che crea in maniera immediata, intuitiva.
\begin{itemize}
\item originale
\item inimitabile
\item intuitivo
\end{itemize}
\pagebreak
festival della mente ogni anno il secondo sabato di settembre.
\end{document}