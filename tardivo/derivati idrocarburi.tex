\documentclass{article}

\usepackage{tikz}
\usepackage{parskip}
\usepackage{amssymb}
\usepackage[utf8]{inputenc}
\usepackage{amsmath, empheq}
\usepackage[margin=3.75cm]{geometry}
\definecolor{mycolour}{RGB}{46,52,64}
\pagecolor{mycolour}
\color{white}

\title{\jobname}
\author{Eugenio Animali}

\begin{document}
\maketitle

\section{Alogenoderivati}

\subsection{Con Alcani}

$CH_4 + Cl_2\to CH_3Cl + HCl$

Con calore, viene separata la molecola del dicloro, creando un radicale: gli elettroni condivisi ritornano ugualmente ai due clori. Un cloro strappa un idrogeno all'idrocarburo, creando un'altro radicale, che si attacca all'altro cloro.

\subsection{Con Alcheni (addizione elettrofila)}

$CH_2=CH_2 + HCl \to CH_3-CH_2Cl$

Si scinde il legame tra H e Cl, lasciando tutti gli elettroni al Cloro, creando due ioni. Si scinde il doppio legame dell'alcheno, per accettare l'idrogeno. Diventa carico negativamente il Cloro, che va ad attaccarsi al carbonio senza legame.

$CH_2=CH_2+Cl_2\to CH_2ClCH_2Cl$

\subsection{Regola di Markovnikov}

Quando un reagente assimetrico si addiziona a un alchene asimmetrico, la componente elettrofila si unisce all'atomo di carbodio a sua volta legato al maggior numero di atomi di idrogeno.

\end{document}