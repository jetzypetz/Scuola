\documentclass{article}

\usepackage{tikz}
\usepackage{parskip}
\usepackage{amssymb}
\usepackage[utf8]{inputenc}
\usepackage{amsmath, empheq}
\usepackage[margin=3.75cm]{geometry}
\definecolor{mycolour}{RGB}{46,52,64}
\pagecolor{mycolour}
\color{white}

\title{\jobname}
\author{Eugenio Animali}

\begin{document}
\maketitle

\section{Cosa é la chimica organica?}
La branca chimica che si occupa di tutti quei composti che si trovano in corpi organici.

Tutti i composti organici hanno uno scheletro di carbonio. 

\begin{enumerate}
    \item Ha una elettronegativitá media - fa legami poco polari, quindi stabili.
    \item É in grado di ibridare gli orbitali - puó promuovere un elettrone della orbitale 2s alla 2p per distribuirli piú uniformemente.
\end{enumerate}
\subsection{Ibridazione}
Puó 'scecherare' gli elettroni per cambiare l'organizzazione delle orbitali per creare:
\begin{enumerate}
    \item 4 orbitali uguali. $sp^3$ forma tetraedrica con orbitali a forma pera doppia sbilanciata. Fa 4 legali singoli ($\sigma$)
    \item 3 orbitali uguali e uno p normale. $sp^2$ Triangolo. Fa 3 legami $\sigma$ a cui si aggiunge la $p$ per renderne uno $\pi$.
    \item 2 orbitali uguali creati tra una s e una p. $sp$
\end{enumerate}
\section{Isomeria}
\begin{enumerate}
    \item Costituzionale
    \begin{enumerate}
        \item Di catena
        \begin{enumerate}
            \item Linearmente
            \item la catena si puó ramificare
        \end{enumerate}
        \item Di posizione (di sostituenti o gruppi funzionali)
        \item Di gruppo (stessi atomi ma sono composti in modo diverso)
    \end{enumerate}
    \item Configurazionali
    \begin{itemize}
        \item Diastereoisomeri = isomeri non sovrapponibili fra loro e che non solo l'immagine speculare 'uno dell'altro.
        \begin{itemize}
            \item isomeri geometrici $\to$ No centri chirali (qualunque atomo di carbonio che lega 4 gruppi diversi)
            \begin{itemize}
                \item cis- i sostituenti puntano entrambi verso l'alto/basso
                \item trans- i sostituenti puntano in direzioni diverse
            \end{itemize}
            \item isomeri ottici $\to$ presenza di piú centri chirali
            sono non sovrapponibili e non speculari fra loro che presentano piú di un centro chirale e hanno configurazione opposta solo in alcuni degli stereocentri
        \end{itemize}
        \item Enantiomeri = isomeri non sovrapponibili fra loro e che sono l'immagine speculare l'uno dell'altro.
    \end{itemize}
    \item Reattivitá delle molecole organiche
    
    quando il Carbonio si lega ad atomi molti elettronegativi, vi é una polaritá, che attrae attacchi da parte di ioni attorno alla molecola.

    meccanismi
    \begin{enumerate}
        \item omolitico - gli elettroni sono spartiti equamente tra gli atomi della scissione
        \item eterolitico - un atomo si prende un elettrone in piú, creado due ioni.
    \end{enumerate}
\end{enumerate}
Chirale- ruota la luce polarizzata
\end{document}