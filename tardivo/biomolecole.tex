\documentclass{article}

\usepackage{tikz}
\usepackage{parskip}
\usepackage{amssymb}
\usepackage[utf8]{inputenc}
\usepackage{amsmath, empheq}
\usepackage[margin=3.75cm]{geometry}
\definecolor{mycolour}{RGB}{46,52,64}
\pagecolor{mycolour}
\color{white}

\title{\jobname}
\author{Eugenio Animali}

\begin{document}
\maketitle
\section{Introduzione}

Le biomolecole sono tutte le molecole che sono utili agli organismi viventi. Ciascun macrogruppo di biomolecole ha una forma generale specifica, caratterizzata da diversi gruppi funzionali, che ne definiscono le qualitá reattive,e vengono alterati nella denaturazione, creando diverse forme.

\section{Caratteristiche principali}

La maggior parte delle molecole, apparte i lipidi, sono polimeriche.

\section{Formazione di un Polimero}

La formazione di un polimero si ha attraverso la condensazione, nella quale due gruppi funzionali reagiscono, producendo acqua, e unendo i monomeri.

\begin{itemize}
\item construzione $\to$ reazione di condensazione
\item Scomposizione $\to$ idrolisi, che necessita acqua per scindere i legami del polimero
\end{itemize}

\section{Carboidrati}

Costituiscono l'impalcatura per tutti i ruoli strutturali. Si chiamano cosi perché formati esclusivamente da C,O,H.

\begin{gather*}
C_n(H_2O)_n\text{ con $n\leq 3$}
\end{gather*}

\subsection{Funzioni}

\begin{enumerate}
    \item Energetica
    \begin{itemize}
    \item Autotrofi: Sintetizzano i zuccheri da $CO_2$ e $H_2O$
    \item Eterotrofi: li utilizzano per ricavare energia
    \end{itemize}
    \item Strutturale
    \begin{itemize}
        \item Negli animali nell'esoscheletro degli invertebrati (chitina)
        \item parete cellulare delle piante (cellulosa)
    \end{itemize}
\end{enumerate}

\subsection{Tipi}

\begin{itemize}
\item Monosaccaridi e Oligosaccaridi (zuccheri)
\item Polisaccaridi (Carboidrati complessi)
\end{itemize}

\subsection{Monosaccaridi}

Hanno normalmente una lunghezza compresa tra 3 - 6(Triosi, Tetrosi, Pentosi, Esosi) atomi di Carbonio. Hanno sempre un gruppo Carbonilico, che quindi li divide in Aldosi e Chetosi (Aldoesoso, Chetoesoso, etc.). Oltre al gruppo Carbonilico, hanno un gruppo Ossidrile su ogni Carbonio.

\subsubsection{Rappresentazione}

Possono essere posti in verticale, con il gruppo Carbonile piú in alto possibile, oppure se é un ciclo, il gruppo Carbossilico creato dalla reazione del Carbonile e un Idrossido va posto piú in alto.

\begin{itemize}
\item Serie D: lo stereocentro piú lontano dal Carbonile ha l'Idrossido a destra.
\item Serie L: lo stereocentro piú lontano dal Carbonile ha l'Idrossido a sinistra.
\end{itemize}

Nei processi metabolici, viene solo usato il D-glucosio. Per il corpo, il L-glucosio é come l'acqua in quanto non interagisce per niente con il corpo.
h
\end{document}