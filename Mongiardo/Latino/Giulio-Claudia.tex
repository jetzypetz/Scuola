\documentclass{article}

\usepackage{tikz}
\usepackage{parskip}
\usepackage{amssymb}
\usepackage[utf8]{inputenc}
\usepackage{amsmath}
\usepackage[margin=3.75cm]{geometry}
\definecolor{mycolour}{RGB}{46,52,64}
\pagecolor{mycolour}
\color{white}

\title{\jobname}
\author{Eugenio Animali}

\begin{document}
\maketitle

\section{Storia della Dinastia Giulio-Claudia}

\subsection{Nascita di Tiberio e la Dinastia Giulia-Claudia}
Ottaviano non designa un erede al trono imperiale per dopo la sua morte. Infatti le tensioni tra senato e Augusto significavano che se avesse tentato di imporre una diretta successione dinastica, avrebbe fatto la stessa fine di Giulio Cesare. Per questo decide di assegnare a Tiberio, figlio adottivo, solo alcuni poteri ed un'ereditá privata. In seguito il senato é forzato a proclamarlo Imperatore per la necessitá di un mediatore tra Senato, esercito, le province ed il popolo romano. Deve essere scelto Tiberio poiché in vita Augusto é riuscito ad affermare quei valori famigliari della antica republica, tra i quali il valore dell'ereditarietá del prestigio.

\subsection{Come diventa Tiberio figlio adottivo di Augusto?}
Tiberio é figlio (insieme a Druso Maggiore) di Tiberio Claudio Nerone, un pretore della gens Claudia, e Livia Drusilla, che in seconde nozze sposa Ottaviano Augusto. Dopo che muoiono i due figli ed i due mariti di Giulia e Druso Maggiore, Tiberio sposa Giulia. Allora ad Augusto non rimane altra scelta che di adottare il figliastro e genero Tiberio, per dargli la possibilitá di diventare erede. Augusto non lo nomina erede per paura di cadere nello stesso errore di Giulio Cesare, che é stato assassinato dal senato per paura. Fa affidamento invece su i valori repubblicani che ha affermato nel suo regno.

\subsection{Inizio dell'Impero Tiberii}
Tiberio proviene da una famiglia conservatrice e quindi mantiene un buon rapporto con il senato all'inizio grazie alle sue idee republicane.

\subsection{Perché Tiberio é conosciuto come l'imperatore paranoico?}
Tiberio ha un figlio Druso Minore che ha la successione dinastica, ma si sentono voci che l'esercito voglia Germanico (figlio di Druso Maggiore) come erede per le sue abilitá militari in quanto generale. Germanico muore misteriosamente, Tiberio é accusato di omicidio. Inizia una terrore, uccidendo chi lo accusa in processi pubblici per \textit{lesa maestá}, e per proteggersi dalle congiure che lui si aspetta, dopo la morte del figlio, scappa a Capri. A Roma lascia il prefetto del pretorio (guardie del corpo- pretoriani, inventate da Augusto), Seiano, nelle sue veci che peró é l'unico che si interessa veramente di utilizzare a suo favore il potere dato. Seiano lavora per cambiare l'ereditá e i famigliari di Tiberio lo informano di queste azioni. Tiberio condanna a morte Seiano.

\subsection{Caligola}
Tiberio sceglie di dare l'ereditá al figlio di Germanico, Gaio Cesare soprannominato Caligola per i sandali che usava mettere anche da piccolo. Lo sceglie per accontentare il popolo ancora memore della morte di Germanico. Caligola é descritto come un pazzo, perché tutti gli storici del tempo sono filosenatorii. Infatti si ricorda la storia del cavallo di Caligola che viene assunto come Governatore. Caligola pretende il culto divino (stampo orientale), che il senato non puó accettare e tenta di instaurare una autocrazia. La politica economica di Caligola consiste nell'uccidere spregiudicatamente parenti e amici, per poi appropriarsi delle loro ricchezze e regalarle al popolo, in grandi opere edilizie ed ai pretoriani. Dopo diverse congiure, viene assassinato Caligola dai pretoriani nel IV anno di impero e nominano il successore Claudio fratello di Germanico zio di Caligola.

\subsection{Claudio}
Claudio é un erudito scrittore. Da la cittadinanza romana a delle province orientali ed estende l'impero nel nord con la bretagna. Fa un grande lavoro di migliorare l'efficienza dell'apparato burocratico, mettendo in posizioni importanti diversi liberti, che spesso diventano piú ricchi di molti senatori. Il suo problema é che lui é un debole, che si fa guidare sia dai liberti che lui mette nell'apparato burocratico, sia dalle mogli.

\subsection{Le Mogli di Claudio}
Claudio si sposa in prime nozze con Messalina, donna di facili costumi, che gli da giá due figli Britannico e Ottavia tra cui Britannico eredabile. Seguentemente si fa manipolare da Agrippina Minore che diventa la IV moglie.
Agrippina ha giá un figlio Nerone, avuto dal suo primo matrimonio con Domizio Enobarbo. Allora Agrippina riesce a fare diseredare a Claudio il figlio di primo letto Britannico, per far riconoscere come legittimo erede Nerone, facendolo sposare con Ottavia per legittimare la successione. Claudio muore improvvisamente, forse avvelenato da Agrippina insieme al prefetto del pretorio Burro.

\subsection{Nerone l'Autocrate}
Nel primo quinquennio, Nerone viene guidato da Seneca, Burro e la madre. Poi uccide la madre, ed elimina progressivamente tutte le persone che in qualche modo vogliono influenzare il suo potere. Burro muore in circostanze poco chiare e viene sostituito da Tigellino, ostile a Seneca, che viene spinto ad abbandonare la politica.

\subsection{Le Riforme}
Furono canbiate le monete per favorire i piccoli risparmiatori. Furono represse ribellioni in Bretagna e in Palestina.

\subsection{L'Incendio}
Viene accusato dagli storici (filosenatori) di iniziare un grande fuoco a Roma nel 64 per liberare una zona per la sua Domus Aurea. Per spostare da se l'accusa, usa i cristiani come capro espiatorio, iniziando la prima persecuzione dei cristiani. Gli storici moderni sono quasi sicuri che non possa essere stato veramente colpevole per l'esercito.

\subsection{Le Congiure}
Poco dopo iniziano diverse congiure per provare a destituire l'Imperatore. La piú importante é quella che vuole mettere al trono Pisone. Viene repressa, e molti congiurati vengono giustiziati o costretti al suicidio. Viene sospettato anche Seneca, a cui viene ordinato di togliersi la vita. Inizia un periodo di terrore, che suscita ribellione da tutte le province. Nerone viene destituito quando Galba, voluto come imperatore dalle truppe spagnole, vince il volere del Senato. Abbandonato addirittura dai pretoriani, Nerone si toglie la vita.

\subsection{Gli Intellettuali nella Dinastia}
A differenza dei circoli letterari organizzati da Augusto, nessuno degli imperatori giulio-claudi riesce a creare un movimento letterario unito e dedicato alla propaganda imperiale, ma si riprende la attenzione, propria di Augusto, di reprimere alcun movimento letterario contrario alla loro politica, per paura di dissenso provocato dallo stesso. Tiberio e Caligola bruciano vivi diversi letterati e opere letterarie che parlano male anche leggermente della tirannia. Non si ricorda alcuna violenza da parte di Claudio, che ama la letteratura. La condanna di Seneca non ha a che fare con la sua produzione letteraria. Gli Imperatori stessi invece sono colti oratori, seguendo la tradizione repubblicana secondo la quale i detentori del potere devono essere uomini colti ed intelligenti.

\subsection{Gli Intellettuali durante l'Impero di Nerone}
Diversa é la scena letteraria con Nerone, che fiorisce, particolarmente nel genere bucolico. É l'unico della dinastia Giulio-Claudia che prova a creare un consenso letterario propagandistico ad imitazione di Augusto. Scrive pure, con aiuto da parte di giovani poeti, un'opera epica, \textit{Troica}, che racconta la storia di Roma. Instituisce pure dei ludi di stampo greco: i Neronia, che comprendono concorsi sportivi, di musica, di poesia e di eloquenza, nei quali gioca pure lui stesso, e vince immancabilmente.

La opposizione filosofica a Nerone é tendenzialmente stoica, come dimostrano i casi di Seneca e Lucano. La opposizione é maggiormente mossa dall'odio per la rigida tirannide e segue il modello di Catone Uticense, considerato il perfetto stoico per la sua rigorosa morale, e impaviditá nei confronti della morte. Infatti sia Seneca che Lucano utilizzarono il pretesto stoico per dare valore ideologico alla loro morte, sebbene fosse ordinata dal princeps.

\section{Fedro}
Fedro nasce in Macedonia, e viene portato a Roma come schiavo da giovane per poi diventare liberto per volere di Augusto grazie alla sua cultura, che lo porterá alla professione di insegnante. Per aiutarsi nella professione, Fedro scrive delle favole, riprendendo quelle di Esopo e traducendole in versi senari giambici in latino. Queste non sono molto amate dai romani, come lui avrebbe pensato, perché é un genere nuovo a Roma; tant'é che viene invece perseguitato da Seiano, ministro di Tiberio, per il carattere satirico di alcuni componimenti in un periodo di repressione e paranoia letteraria da parte degli imperatori. Se ne lamenta nel prologo del III di V libri, dopo la morte di Seiano nel 31 d.C. (importante data per la cronologia delle opere). Verso la fine del periodo antico, e il medioevo, le favole di Fedro divennero molto diffuse nell'insegnamento del latino, ma sempre in forma di copie in prosa anonima; finché nel XVI secolo fu riscoperta l'opera originale.
\subsection{Il Modello e il genere}
Fedro si ispira, come spega nel primo prologo, ad Esopo, antico personaggio leggendario; schiavo dell'Asia Minore che raccolse in forma letteraria le storielle che si tramandavano oralmente in Magna Grecia. Esopo é quindi l'iniziatore del genere favola, che diventa una forma semplice e distolta (i personaggi sono spesso animali e piante parlanti), dove velare forti critiche ai potenti, che opprimono gli umili come lui stesso, nelle morali di storielle innocenti. Lo scrittore di favole sta quasi insegnando in maniera condiscendente le basi dell'educazione ai suoi superiori. 

Notiamo un parallelismo tra la favola e la commedia:
\begin{enumerate}
    \item Metro: I dialoghi delle commedie, come queste favole, sono scritti in senari giambici.
    \item Intento: Fedro stesso ci spiega che intende \emph{risum movere}, proprio come le commedie.
    \item Carattere: Rappresentano entrambi situazioni umili e realistiche.
    \item Stile: In entrambi vi é una impostazione drammatica in cui due o piú personaggi discutono.
\end{enumerate}
Pare infatti che la scelta del metro sia intenta ad avvicinare la nuova forma della favola alla commedia. Scrittori di favole precedenti sono i satiristi Ennio, Lucilio e Orazio che scrivevano in esametri.
\subsection{I contenuti e le caratteristiche dell'opera}
Sia nel primo sia nel secondo prologo, afferma che come Esopo, lui ha un duplice scopo:
\begin{enumerate}
    \item divertire; dar piacere all'orecchio
    \item \emph{monere} o consigliare; correggere gli errori degli uomini
\end{enumerate}
Nel secondo prologo, chiede a Esopo se gli possa essere concesso di aggiungere qualcosa di suo (\emph{aliquid interponere}) per dilettare il lettore con un poco di \emph{varietas}. In cambio della benevolenza che chiede al suo pubblico, egli assicura che continuerá ad attenersi al criterio della \emph{brevitas}. La \emph{varietas} e la \emph{brevitas} quindi diventano le basi della poetica fedriana. 
\subsubsection{Varietas}
Iniziando dal primo libro infatti, vediamo Fedro iniziare ad includere anche personaggi come Esopo, Socrate, e divinitá dell'Olimpo; e storielle piú realistiche oltre agli animali parlanti del modello. La originalitá si dimostra anche nelle aggiunte di anedddoti storici (di ambientazione romana). É cosí che Fedro attenua la sua \emph{aemulatio}.
\subsubsection{Brevitas}
Fedro si occupa della concisione nell'estensione delle storie e dei libri; quanto alla capicitá di condensare i dialoghi e gli insegnamenti morali, per ottenere l'interesse del pubblico inabituato:
\begin{enumerate}
    \item storia di gatti che portano un gallo in lettiga, il quale é tutto fiero, ma la volpe gli dice di stare attento perché i gatti lo vogliono mangiare.
    \item Montagna che partorisce un topo. ``Questo ho scritto per te, che fai grandi minacce, ma poi alla fine non concludi nulla.''
\end{enumerate}
\subsection{Morale}
Il punto di vista della morale é degli umili. Intende mostrare la vita e i comportamenti degli uomini per protestare indirettamente alla situazione sociale.
 É pur sempre pessimistica la sua visione dei prepotenti e dell'abbandono alla propria posizione che l'umile deve accettare: si vede nella favola dell'asino, che trova uguale essere oppresso da un prepotente o un altro. Fedro inserisce spunti diatribici come l'affermazione del valore della libertá.
\subsection{Il Lupo e L'agnello}
Il piú forte ha sempre la meglio sul piú debole. É una critica alla ipocrisia di cercare una scusa per presentarsi come giusto.
\subsection{L'asino}
É uguale essere sottomessi da un capo che un'altro.
\subsection{Proemio}
Queste favole sono state inventate proprio per denunciare i capi senza essere repressi.
\subsection{La volpe e l'uva}
\section{Seneca}
\subsection{Vita}
Lucio Anneo Seneca nasce a Cordova verso il 4 a.C. figlio di ricca famiglia. Come ci spega in delle Epistole a Lucilio, viene mandato subito a Roma a studiare con diversi precettori, tra cui Sozione, che gli insegna ad allontanarsi dai piaceri come il vino, le delizie gastronomiche ed i bagni caldi (usa addirittura fare il bagno nel tevere per inaugurare l'anno nuovo), per diventare una persona piú resistente al dolore.

I suoi precettori, come lui, vogliono che si dedichi a vita privata ma lui segue il volere del padre, e intraprende un \textbf{cursus honorum}, diventando un abile questore. La sua fama non viene condivisa dagli imperatori, che hanno tutti interesse ad eliminarlo. Caligola lo vuole uccidere, ma convinto da una gentildonna che Seneca sia terminalmente malato, lo lascia vivere. Claudio lo accusa di adulterio per volere della prima moglie Messalina, che vuole mettere in cattiva luce Giulia Livilla, sorella di Caligola; Seneca viene mandato in esilio in Corsica per essere poi richiamato dalla IV moglie Agrippina, che lo vuole come precettore del figlio Nerone. Nel 54 muore Claudio e Seneca sale al potere come reggente di Nerone, ancora minorenne. Del principe si occupano Seneca, la madre Agrippina e il prefetto del pretorio Burro. Seneca spera di far crescere in Nerone un imperatore esemplare, come esprime nel \emph{De Clementia}, ma invece Nerone diviene un dispote sfrenato, che uccide la madre per le sue intromissioni nel potere. Seneca ne fa sicuramente parte, ma deve ingoiare il rospo per non diventare un nemico del \textbf{princeps}. Quando nel 62 muore il moderato Burro e viene sostituito dal severo Tigellino, Seneca si ritira a vita privata con la scusa di etá e salute.

Gli anni seguenti sono i piú goduti di Seneca, perché si puó finalmente dedicare alla vita contemplativa che ha sognato sin dalla giovinezza. Pensa, studia e scrive; finché non viene scoperta la congiura di Pisone.

Seneca é considerato complice e costretto a togliersi la vita. É affiancato da Paolina che tenta, eroina stoica, di morire insieme al marito, ma ne é impedita dai pretoriani di Nerone. Tacito raffigura questo suicidio come il suicidio modello stoico per la libertá. Suicidio virtuoso; Seneca consola gli altri: non ho paura della morte, come ho scritto in tante pagine. Nerone, dopo aver ucciso madre e fratello, non poteva che uccidere anché il suo educatore. Seneca chiede alla moglie di non uccidersi ma lei dice che la aspetta la morte. Allora Seneca é d'accordo perché lei avrá piú gloria in quanto non costretta. Nerone non lascia che lei si uccida perché accrescerebbe la risposta negativa. Seneca soffre e chiede il veleno, anche se é gia quasi morto. Viene portato in un bagno dove muore soffocato.
\subsection{\emph{Apokolokyntosis} o \emph{Ludus de morte Claudii}}
\subsubsection{Genere}
É una satira menippea:
\begin{enumerate}
    \item Forma: mescolanza di versi e prosa
    \item Contenuto: commistione id serio e scherzoso
\end{enumerate}
...in forma di \textbf{pamphlet}
\subsubsection{Interpretazioni del Titolo}
\begin{enumerate}
    \item \textbf{Apotheosis} (trasformazione in dio) + \textbf{Kolokynte} (zucca) =\textbf{Apokolokyntosis} (tras- formazione in zucca) ma non vi é alcuna trasformazione...
    \item Deificazione di una zucca
    \item Espressione idiomatica perduta? infinocchiatura/ fregatura
\end{enumerate}
\subsubsection{Storia}
Finalmente le parche tagliano il filo della vita di Claudio e Apollo canta per l'arrivo del nuovo imperatore Nerone. Va in cielo e non viene accettato da Giove perché parla in modo incomprensibile (perché non parlava bene). Caricatura del personaggio. Non si capisce se é uomo o animale ed Ercole si prepara alla sua tredicesima fatica. Gli dei capiscono e fanno un concilio per decidere se accettare Claudio. Segue un dibattito, finche dio Augusto inizia a parlare, accusandolo di aver assassinato parenti e viene mandato agli inferi e solo quando vede il suo funerale capisce di essere morto (stupido). Vede tutti che festeggiano sulla via Sacra. Gli uccisi vogliono vendetta su di lui (processi finti). Condannato a giocare ai dadi con un bussolotto forato (pena stupida per un uomo stupido). Caligola lo vuole schiavo. Non viene consegnato a Caligola ma ad un liberto (era stato manipolato da liberti).
\subsubsection{t14}
`mi sono smerdato' dissacrazione del topos del morente che parla, che dovrebbe dire cose importanti, invece Seneca gli fa dire una cosa banale e aggiunge: Non so se l'abbia detto veramente, ma so che ha smerdato l'impero.
\subsubsection{t1 (De Brevitate Vitae)}
Seneca afferma che non é vero la vita sia breve (\textbf{tam velociter, tam rapide}), ma sono gli stolti che non la vivono, perché si fanno portare dalle delizie ed il godere. Il saggio vive esaustivamente ogni momento e la vita gli sembra molto piú lunga. 

`\textbf{mortales}' utilizzato perché (1) é piú rara, (2) entra subito in argomento per sottolineare che moriremo tutti.

(1) Ne si lamentano, come pensano, soltanto la turba e l'ignaro popolo di questo male, che é di tutti: questa condizione evocava anche le lamentele dei uomini intelligenti. Da qui viene quella esclamazione del piú importante dei poeti:``la vita é breve, ed é lunga l'arte''

(3) Non é esiguo il tempo che abbiamo ma é molto quello che perdiamo. É data una vita abbastanza lunga e ampia in c-fino a sumus

\section{Dialogi}
Compendio di 10 opere che include la \textit{De Brevitate Vitae}. Dialogo tra l'autore Seneca e il destinatario dell'opera. Sono quindi dei monologhi perché le risposte degli interlocutori sono brevi e insignificanti solo per mantenere la cornice dialogica. Uso terapeutico della filosofia: funzione é di insegnare l'interlocutore a vivere felice, rispondendo in ciascun dialogo ad un diverso aspetto delle paure della vita (e.g. temere la morte, il male che succede ai buoni).

\subsection{Consolationes}
Consola diverse persone con la filosofia stoica come terapia. Spesso mente, poiché vediamo dopo che la sua opinione é opposta.

\subsubsection{Consolazio ad Marciam}
Il testo tenta di consolare la donna Marcia che ha perso un figlio, spiegando che la morte non é un male, e che il ragazzo avrá una vita migliore nell'al di lá.

\subsubsection{Consolatio ad Helviam matrem}
Consola la madre di Seneca. Spiega che non é cosi male, anche se Seneca non é contento e quando torna scrive l'Apokolokyntosis per sfogarsi.

\subsubsection{Consolatio ad Polybium}
Consola un liberto claudii che ha perso un fratello.

\subsection{Dialoghi-Trattati}

\subsubsection{De Brevitate Vitae}
La vita non é breve, solo che lo stolto la riempie di futili occupazioni come il collezionismo. Bisogna solo occuparsi del benessere dell'anima con lo studio e la meditazione. ``Vita, si uti scias, longa est'' longa est= apodosi in prima. Scias= protasi in seconda della possibilitá perché congiuntivo.

\subsubsection{t4}
Solo introduzione

\subsubsection{t5}
In Italiano

\subsection{De Ira}

\section{De clementia}
Anch'essi sono dialoghi ma tramandati da fonti diverse.

Con Augusto, era ancora una repubblica, e Augusto rivestiva semplicemente tutte le cariche. Se si parla di Clemenza, significa che é una monarchia assoluta. Un rex iussus deve essere clemente, e deve non infierire. Utilizzare la violenza solo quando é necessaria. Rapporto paterno con il popolo: punire i figli solo quando serve. Opera scritta all'inizio del impero di Nerone. Non vi é la ipocrisia Augustea, sa di essere scelto dagli dei.

Viene presentato come Nerone che fa un discorso clemente. Autoelogio.

\end{document}