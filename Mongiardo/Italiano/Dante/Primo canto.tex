\documentclass{article}

\usepackage{parskip}
\usepackage[utf8]{inputenc}
\usepackage{amsmath, empheq}
\usepackage[dvipsnames]{xcolor}
\usepackage[margin=3.75cm]{geometry}
\definecolor{mycolour}{RGB}{46,52,64}
\pagecolor{mycolour}
\color{white}

\title{Primo canto}
\author{Eugenio Animali}
\date{15 09 2022}

\begin{document}

\maketitle

\section*{Proemio}

\begin{itemize}
\item Livello si é innalzato perché vi é un proemio
\item vv 1-36
\item Paradiso dedicato a Cangrande della Scala- epistola a Cangrande spiega il nome.
\item Mezzogiorno dell'equinozio di primavera (13 aprile 1300)
\end{itemize}

\section*{Parafrasi}
La gloria (simbolo é luce) di Dio (primo motore immobile) risplende (penetra) ovunque, ma piu o meno secondo la zona.

Io sono stato nel cielo dove risplende di piú la sua luce; e ho visto cose che non saprei ridire (memoria- non si ricorda, linguaggio- non troverebbe comunque le parole).

Perché avvicinandosi a Dio, la nostra mente si immerge tanto che la memoria non non riesce ad andare cosí profondamente (parti dell'intelletto separate - da Cavalcanti).

Quello che ricordo qui scriveró.

O Apollo, per l'ultima cantica, fai di me un vaso (riempimi) del tuo valore allorche possa meritare l'alloro(non l'aveva ricevuto perché a roma c'era Bonifacio VIII).

Fino a qui mi é stato sufficente un gioco di Parnaso (le muse) ma ora con entrambe le cime (apollo e le muse) m'é necessario (latinismo - opus est) per entrare nel campo di battaglia.

Ispirami tu come quando traesti fuori Marsia (satiro) dalla pelle delle sue membra.

O virtú divina, se mi aiuti tanto che io possa manifestare l'ombra del bel paradiso che mi é rimasta in memoria, mi vedrai venire ai piedi del tuo amato albero (d'alloro) e farmi una corona in foglie, di cui la materia (paradiso) e l'aiuto tuo mi faranno degno.

Cosí raramente si raccolgono foglie di alloro, o apollo, per il trionfo di un cesare o un poeta; perche gli uomini non puntano alla gloria, ma al potere e i soldi

che l'alloro (di Peneo), quando qualcuno la desidera, dovrebbe infondere gioia nei confronti di Apollo (lieto, delfico - oracolo di delfi, deitá).(falsa modestia)

Una piccola fiamma (Dante) sará seguita da una piú grande - qualcunaltro fará un  giorno una migliore invocazione ad Apollo.

\section*{Storia}
Beatrice guarda il sole di mezzogiorno e dante la imita. Iniziano a salire e sente la musica celeste. Dante si chiede come fa a volare.(vv.98-99 Ma ora ammiro con'io trascenda questi corpi lievi) e Beatrice gli spiega che si é tanto purificato dal percorso da renderlo leggero e inclinato naturalmente verso dio, non piú trascinato giu dal peccato.
\section*{Temi}
Non ci basta di invocare le muse... deve invocare il dio della poesia, Apollo. Non é un problema l'invocazione pagana (accettato per scopo letterario). "Amato Alloro"- riferimento alle metamorfosi di Ovidio, Apollo ama Dafne, trasformata da padre Peneo.
Marsio sfida Apollo a suonare il flauto di pan contro il cetro. Perde e per punire la superbia com'é solito degli dei sfidati, lo appende ad un albero e lo spella vivo.
Questi riferimenti sono un monito a se stesso a rimanere umile, in una situazione facilmente compresa come superbia, peccato di cui Dante ha molta paura.
\end{document}